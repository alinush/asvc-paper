An \textit{aggregatable subvector commitment (aSVC)} scheme is a \textit{vector commitment (VC)} scheme that can aggregate multiple proofs into a single, small subvector proof.
In this paper, we formalize aSVCs, give an efficient construction in prime-order groups from constant-sized polynomial commitments and use it to bootstrap a highly-efficient stateless cryptocurrency.
\\

Our aSVC supports (1) computing all $n$ $O(1)$-sized proofs in $O(n\log{n})$ time, (2) updating a proof in $O(1)$ time and (3) aggregating $b$ proofs into an $O(1)$-sized subvector proof in $O(b\log^2{b})$ time.
Importantly, our scheme has an $O(n)$-sized proving key, an $O(1)$-sized verification key and $O(1)$-sized update keys.
In contrast, previous schemes with constant-sized proofs in prime-order groups either (1) require $O(n^2)$ time to compute all proofs, (2) require $O(n)$-sized update keys to update proofs in $O(1)$ time, or (3) do not support aggregating proofs.
Furthermore, schemes based on hidden-order groups either (1) have larger concrete proof size and computation time, or (2) do not support proof updates.
\\

We use our aSVC to obtain a stateless cryptocurrency with very low communication and computation overheads.
Specifically, our constant-sized, aggregatable proofs reduce each block's proof overhead to just one group element, which is optimal.
In contrast, previous work required $O(b\log{n})$ group elements, where $b$ is the number of transactions per block.
Furthermore, our smaller proofs reduce the block verification time from $O(b\log{n})$ pairings to just two pairings and an $O(b)$-sized multi-exponentiation.
Lastly, our aSVC's smaller update keys only take up $O(b)$ block space, compared to $O(b\log{n})$ in previous work.
Also, their zverifiability reduces miner storage from $O(n)$ to $O(1)$.
The end result is a stateless cryptocurrency that concretely and asymptotically outperforms the state of the art.
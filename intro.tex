%
\section{Introduction}

\blfootnote{An errata for this paper can be found at \url{https://github.com/alinush/asvc-paper}.}
In a \textit{stateless cryptocurrency}, neither \textit{miners} nor cryptocurrency \textit{users} need to store the full \textit{ledger state}.
Instead, this state consisting of users' account balances is split among all users using an \textit{authenticated data structure}.
This way, miners only store a succinct \textit{digest} of the ledger state and each user stores their account balance.
Nonetheless, miners can still validate transactions sent by users, who now include \textit{proofs} that they have sufficient balance.
Furthermore, miners can still propose new \textit{blocks} of transactions and users can easily \textit{synchronize} or \textit{update} their proofs as new blocks get published.

Stateless cryptocurrencies have received increased attention~\condcite{Dryja19,RMCI17,CPZ18,BBF19,GRWZ20}{ST99,Miller12,Todd16,Buterin17} due to several advantages.
First, stateless cryptocurrencies eliminate hundreds of gigabytes of miner storage needed to validate blocks.
Second, statelessness makes scaling consensus via \textit{sharding} much easier, by allowing miners to efficiently switch from one shard to another~\condcite{KJG+18}{EthereumSharding}.
Third, statelessness lowers the barrier to entry for full nodes, resulting in a much more resilient, distributed cryptocurrency.

\parhead{Stateless Cryptocurrencies from VCs.}
At a high level, a VC scheme allows a \textit{prover}  to compute a succinct \textit{commitment} $c$ to a \textit{vector} $\vect{v}=[v_0, v_1, \dots, v_{n-1}]$ of  $n$ \textit{elements} where $v_i\in \Zp$.
%Although we are referring to it as a ``commitment'', we will be more interested in its \textit{binding}, rather than its \textit{hiding}, properties.
Importantly, the prover can generate a \textit{proof} $\pi_i$ that $v_i$ is the element at position $i$ in $\vect{v}$, and any \textit{verifier} can check it against the commitment $c$.
The prover needs a \textit{proving key} $\prk$ to commit to vectors and to compute proofs, while the verifier needs a \textit{verification key} $\vrk$ to verify proofs.
(Usually $|\vrk| \ll |\prk|$.)
Some VC schemes support \textit{updates}: if one or more elements in the vector change, the commitment and proofs can be updated efficiently.
For this, a static \textit{update key} $\upk_j$ tied only to the updated position $j$ is necessary.
Alternatively, some schemes require dynamic \textit{update hints} $\uph_j$, typically consisting of the actual proof $\pi_j$.
The proving, verification and update keys comprise the VC's \textit{public parameters}.
Lastly, \textit{subvector commitment (SVC)} schemes~\cite{LM19} support computing succinct proofs for \textit{$I$-subvectors} $(v_i)_{i\in I}$ where $I\subset [0,n)$.
Furthermore, some schemes are \textit{aggregatable}: multiple proofs $\pi_i$ for $v_i, \forall i \in I$ can be aggregated into a single, succinct $I$-subvector proof.

Chepurnoy, Papamanthou and Zhang pioneered the idea of building \textit{account-based}~\cite{Ethereum}, stateless cryptocurrencies on top of any \textit{vector commitment (VC)} scheme~\cite{CPZ18}.
Ideally, such a VC would have (1) sublinear-sized, updatable proofs with sublinear-time verification, (2) updatable commitments and (3) sublinear-sized update keys.
In particular, static update keys (rather than dynamic update hints) help reduce interaction and thus simplify the design (see \cref{s:stateless-cryptocurrency:edrax}).
We say such a VC has ``\textit{scalable updates}.''
Unfortunately, most VCs do not have scalable updates (see \cref{s:related-work\ifNotCameraReady,t:asvc-comparison,t:rsa-asvc-comparison-appendix\fi}) or, if they do~\cite{CPZ18,Tomescu20}, they are not optimal in their proof and update key sizes.
Lastly, while some schemes in hidden-order groups have scalable updates~\cite{CFG+20}, they suffer from larger concrete proof sizes and are likely to require more computation in practice.

\parhead{Our Contributions.}
In this paper, we formalize a new \textit{aggregatable subvector commitment (aSVC)} notion that supports commitment updates, proof updates and aggregation of proofs into subvector proofs.
Then, we construct an aSVC \textit{with scalable updates} over pairing-friendly groups.
Compared to other pairing-based VCs, our aSVC has constant-sized, aggregatable proofs that can be updated with constant-sized update keys (see \cref{t:asvc-comparison}).
Furthermore, our aSVC supports computing all proofs in quasilinear time.
\ifCameraReady
We prove our aSVC secure under $q$-SBDH~\cite{Goyal07} in the extended version of our paper~\cite{TAB+20e}.
\else
We prove security of our aSVC by strengthening (and re-proving) the security definition of KZG polynomial commitments~\cite{KZG10a}.
\fi

\begin{table*}[t]
    %\large
    %\small
    %\footnotesize
    %\scriptsize
    \centering
    \caption{
        Asymptotic comparison of our work with other stateless cryptocurrencies.
        $n$ is the number of users, $\lambda$ is the security parameter, and $b$ is the number of transactions in a block.
        $\Gexp$ is an \textit{exponentiation} in a known-order group.
        $\mathbb{G}_?$ is a (slower) \textit{exponentiation} (of size $2\lambda$ bits) in a hidden-order group.
        $\mathbb{P}$ is a pairing computation.
        $|\pi_i|$ is the size of a proof for a user's account balance.
        $|\upk_i|$ is the size of user $i$'s update key.
        $|\pi_I|$ is the size of a proof aggregated from all $\pi_i$'s in a block.
        We give each \textit{Miner's storage} in terms of VC public parameters (e.g., update keys).
        A miner takes: (1) \textit{Check digest time}, to check that, by ``applying'' the transactions from block $t+1$ to block $t$'s digest, he obtains the correct digest for block $t+1$,
        % Note: This can be further asymptotically reduced (see Overleaf notes)
        (2) \textit{Aggr. proofs time}, to aggregate $b$ transaction proofs, and
        (3) \textit{Vrfy. $|\pi_I|$ time}, to verify the aggregated proof.
        A user takes \textit{Proof synchr. time} to ``synchronize'' or update her proof by ``applying'' all the transactions in a new block.
        We treat \cite{GRWZ20} and \cite{CFG+20} as a payments-only stateless cryptocurrency without smart contracts.
        Our aggregation and verification times have an extra $b\log^2{b}\Fop$ term, consisting of very fast field operations.
        \ifNotCameraReady A detailed analysis of the underlying VCs can be found in \cref{s:complexity:pointproofs,s:complexity-lagrange-asvc,s:complexity:cpz,s:complexity-bbf}.\xspace\fi
    }
    \label{t:stateless-comparison} % must go after \caption{} for \cref{} to work
    \setlength{\tabcolsep}{.4em} % space between columns
    \begin{tabular}{ccccc}
        %\toprule
        {\makecell{Account-based\\stateless\\cryptocurrencies}}
        & \makecell{Edrax\\\cite{CPZ18}}
        & \makecell{Pointproofs\\\cite{GRWZ20}}
        & \makecell{2nd VC of\\~\cite{CFG+20}}
        & \textbf{Our work}\\
        \toprule
        $|\pi_i|$               & $\rl\Gsz$    & 1\Gsz              & 1\Ghsz                     & 1\Gsz\\
        $|\upk_i|$              & $\rl\Gsz$    & \rn\Gsz            & 1\Ghsz                     & 1\Gsz\\
        $|\pi_I|$               & $\rbl\Gsz$   & 1\Gsz              & 1\Ghsz                     & 1\Gsz\\
        Miner's storage         & $\rn\Gsz$    & $\rn\Gsz$          & 1\Ghsz                     & $b\Gsz$\\
        Vrfy. $|\pi_I|$ time    & $b\rl\Pair$  & $2\Pair+b\Gexp$    & $b\myred{\log{b}}\Ghop$    & $2\Pair+ b\Gexp + \rblb\Fop$\\
        Check digest time       & $b \Gexp$    & $b\Gexp$           & $b\Ghop$                   & $b\Gexp$\\
        Aggr. proofs time       & \nop         & $b\Gexp$           & $b\myred{\log^2{b}} \Ghop$ & $b\Gexp + \rblb\Fop $\\
        Proof synchr. time      & $b\rl\Gexp$  & $b\Gexp$           & $b\Ghop$                   & $b\Gexp$\\
    \end{tabular}
\end{table*}

\paragraph{A Highly-Efficient Stateless Cryptocurrency.}
We use our aSVC to construct a stateless cryptocurrency based on the elegant design of Edrax\xspace\cite{CPZ18}.
Our stateless cryptocurrency has very low storage, communication and computation overheads (see \cref{t:stateless-comparison}).
First, our constant-sized update keys have a smaller impact on block size and help users update their proofs faster.
Second, our proof aggregation drastically reduces block size and speeds up block validation.
Third, our verifiable update keys removes the need for miners to either (1) store all $O(n)$ update keys or (2) interact during transaction validation to check update keys.

\subsection{Related Work}
\label{s:related-work}

\parhead{Vector Commitments (VCs).}
The notion of VCs appears early in~\cite{CFM08,LY10,KZG10a} but Catalano and Fiore~\cite{CF13} are the first to formalize it.
They introduce schemes based on the Computational Diffie-Hellman (CDH), with $O(n^2)$-sized public parameters, and on the RSA problem, with $O(1)$-sized public parameters, which can be \textit{specialized} into $O(n)$-sized ones when needed.
%Neither scheme is aggregatable.
Lai and Malavolta~\cite{LM19} formalize \textit{subvector commitments (SVCs)} and extend both constructions from~\cite{CF13} with constant-sized $I$-subvector proofs.
Camenisch et al.~\cite{CDHK15} build VCs from KZG commitments~\cite{KZG10a} to Lagrange polynomials\ifNotCameraReady\xspace(see \cref{s:prelim:interpolation})\fi\xspace that are not only \textit{binding} but also \textit{hiding}.
However, their scheme intentionally prevents aggregation of proofs as a security feature.
%Their \textit{verification key} (see \cref{s:prelim:vcs}) is $O(n)$-sized and they do not discuss updating proofs.
Feist and Khovratovich~\cite{FK20} introduce a technique for precomputing all \textit{constant-sized} evaluation proofs in KZG commitments when the evaluation points are all roots of unity.
We use their technique to compute VC proofs fast.
Chepurnoy et al.~\cite{CPZ18} instantiate VCs using multivariate polynomial commitments~\cite{PST13} but with logarithmic rather than constant-sized proofs.
Then, they build the first efficient, account-based, stateless cryptocurrency on top of their scheme.
Later on, Tomescu~\cite{Tomescu20} presents a very similar scheme but from univariate polynomial commitments~\cite{KZG10a} which supports subvector proofs.

Boneh et al.~\cite{BBF19} instantiate VCs using hidden-order groups.
They are the first to support aggregating multiple proofs (under certain conditions\ifNotCameraReady; see \cite[Sec. 5.2, p. 20]{BBF18}\fi).
They are also the first to have constant-sized public parameters, without the need to specialize them into $O(n)$-sized ones.
However, their VC uses update hints (rather than keys), which is less suitable for stateless cryptocurrencies.
\ifNotCameraReady
Furthermore, they introduce \textit{key-value map commitments (KVCs)}, which support a larger set of positions from $[0, 2^{2\lambda})$ rather than $[0,n)$, where $\lambda$ is a security parameter.
They argue their KVC can be used for account-based stateless cryptocurrencies, but do not explore a construction in depth.
\fi
% They do not talk about updating KVC proofs.
Campanelli et al.~\cite{CFG+20} also formalize SVCs with a more powerful notion of \textit{infinite (dis)aggregation} of proofs.
In contrast, our aSVC only supports ``one hop'' aggregation and does not support disaggregation.
They also formalize a notion of updatable, distributed VCs as Verified Decentralized Storage (VDS).
However, their use of hidden-order groups leads to larger concrete proof sizes.
% Note: Regarding "and to formalize the notion of \textit{specialization} for public parameters," it's probably more fair to attribute this to CF13.
\ifNotCameraReady
Both of their schemes have $O(1)$-sized public parameters and can compute all proofs efficiently in quasilinear time.
One scheme supports update hints while the other supports update keys.
\fi

Concurrent with our work, Gorbunov et al.~\cite{GRWZ20} also formalize aSVCs with a stronger notion of \textit{cross-commitment aggregation}.
However, their formalization lacks (verifiable) update keys, which hides many complexities that arise in stateless cryptocurrencies (see \cref{s:stateless-cryptocurrency:dos-update-key}).
Their VC scheme extends~\cite{LY10} with (1) aggregating proofs into $I$-subvector proofs and (2) aggregating multiple $I$-subvector proofs \textit{with respect to different VCs} into a single, constant-sized proof.
However, this versatility comes at the cost of (1) losing the ability to precompute all proofs fast, (2) $O(n)$-sized update keys for updating proofs, and (3) $O(n)$-sized verification key.
% Note: For updating comitments, the update key for position $i$ is just one group element.
This makes it difficult to apply their scheme in a stateless cryptocurrency for payments such as Edrax~\cite{CPZ18}.
Furthermore, Gorbunov et al. also enhance KZG-based VCs with proof aggregation, but they do not consider proof updates.
Lastly, they show it is possible to aggregate $I$-subvector proofs across different commitments for KZG-based VCs.

%Libert et al.~\cite{LRY16} generalize VCs to \textit{functional commitments (FCs)} which, instead of revealing $v_i$ when opening, reveals $\sum_{i\in[0,n)} x_i v_i$, for any $\vect{x}=(x_i)_{i\in[0,n)}$ given as input to the opening algorithm.
%Lai and Malavolta~\cite{LM19} generalize FCs to \textit{linear map commitments (LMCs)} which reveals $f(\vect{v})$ for any linear map $f : \Fp^n \rightarrow \Fp^q$ given as input to the opening algorithm ($q$ is fixed at setup).
Kohlweiss and Rial~\cite{KR13} extend VCs with zero-knowledge protocols for proving correct computation of a new commitment, for opening elements at secret positions, and for proving secret updates of elements at secret positions.

% Note: BCFK19 for zero-knowledge sets from CPZ18 VC. Cool, but unrelated.
% Note: Thakur19 claims VC from bilinear accs, but does not give full description.

\parhead{Stateless Cryptocurrencies.}
The concept of stateless validation appeared early in the cryptocurrency community~\cite{Miller12,Todd16,Buterin17} and later on in the academic community~\cite{RMCI17,Dryja19,CPZ18,BBF19,GRWZ20}.
\ifNotCameraReady\paragraph{UTXO-based.}\fi
Initial proposals for UTXO-based cryptocurrencies used Merkle hash trees~\cite{Miller12,Todd16,Dryja19,CPZ18}.
In particular, Dryja~\cite{Dryja19} gives a beautiful Merkle forest construction that significantly reduces communication.
Boneh et al.~\cite{BBF19} further reduce communication by using RSA accumulators\ifNotCameraReady~\cite{Bd93,LLX07}\fi.

\ifNotCameraReady\paragraph{Account-based.}\fi
Reyzin et al.~\cite{RMCI17} introduce a Merkle-based construction for account-based stateless cryptocurrencies.
Unfortunately, their construction relies on \textit{proof-serving nodes}: every user sending coins has to fetch the recipient's Merkle proof from a node and include it with her own proof in the transaction.
% Note: Fundamentally, this is because Merkle trees use update hints $\uph_i$ consisting of the Merkle proof $\pi_i$.
Edrax~\cite{CPZ18} obviates the need for proof-serving nodes by using a vector commitment (VC) with update keys (rather than update hints like Merkle trees).
Nonetheless, proof-serving nodes can still be used to assist users who do not want to manually update their proofs (which is otherwise very fast).
Unfortunately, Edrax's (non-aggregatable) proofs are logarithmic-sized and thus sub-optimal.

Gorbunov et al.~\cite{GRWZ20} introduce \textit{Pointproofs}, a versatile VC scheme which can aggregate proofs across \textit{different} commitments.
They use this power to solve a slightly different problem: stateless block validation for smart contract executions (rather than for payments as in Edrax).
Unfortunately, their approach requires miners to store a different commitment for each smart contract, or around 4.5 GBs of (dynamic) state in a system with $10^8$ smart contracts.
This could be problematic in applications such as sharded cryptocurrencies, where miners would have to download part of this large state from one another when switching shards.
Lastly, the verification key in Pointproofs is $O(n)$-sized, which imposes additional storage requirements on miners.
Furthermore, Gorbunov et al. do not discuss how to update nor precompute proofs efficiently.
Instead they assume that all contracts have $n\le 10^3$ memory locations and users can compute all proofs in $O(n^2)$ time.
In contrast, our aSVC can compute all proofs in $O(n\log{n})$ time~\cite{FK20}.
Nonetheless, their approach is a very promising direction for supporting smart contracts in stateless cryptocurrencies.

Bonneau et al.~\cite{BMRS20} use recursively-composable, succinct non-interactive arguments of knowledge (SNARKs) \cite{BCTV14} for stateless validation.
However, while block validators do not have to store the full state in their system, miners who propose blocks still have to.
\ifNotCameraReady
In contrast, in previous stateless cryptocurrencies (including ours), even miners who propose blocks are stateless.
\fi

\section{A Highly-efficient Stateless Cryptocurrency}
\label{s:stateless-cryptocurrency}

In this section, we enhance Edrax's elegant design by replacing their VC with our secure \textit{aggregatable} subvector commitment (aSVC) scheme from \cref{s:asvc:from-kzg}.
As a result, our stateless cryptocurrency has smaller, aggregatable proofs and smaller update keys.
This leads to smaller, faster-to-verify blocks for miners and faster proof synchronization for users (see \cref{t:stateless-comparison}).
Furthermore, our verifiable update keys reduce the storage overhead of miners from $O(n)$ update keys to $O(1)$.
We also address a denial of service (DoS) attack in Edrax's design.

\subsection{From VCs to Stateless Cryptocurrencies}
\label{s:stateless-cryptocurrency:edrax}
Edrax pioneered the idea of building account-based, stateless cryptocurrencies on top of any VC scheme~\cite{CPZ18}.
In contrast, previous approaches were based on \textit{authenticated dictionaries (ADs)}~\cite{RMCI17,Buterin17}, for which efficient constructions with static update keys are not known.
In other words, these AD-based approaches used \textit{dynamic update hints} $\uph_j$ consisting of the proof for position $j$.
This complicated their design, requiring user $i$ to ask a \textit{proof-serving node} for user $j$'s proof in order to create a transaction sending money to $j$.

\parhead{Trusted Setup.}
To support up to $n$ users, public parameters $(\prk,\vrk$, \ifCameraReady\linebreak[4]\fi$(\upk_i)_{i\in[0,n)})\leftarrow \vcsetup(1^{\lambda}, n)$ are generated via a \textit{trusted setup}, which can be decentralized using MPC protocols~\cite{BGM17}.
Miners need to store all $O(n)$ update keys to propose blocks and to validate blocks (which we fix in \cref{s:stateless-cryptocurrency:dos-update-key}).
The $\prk$ is only needed for \textit{proof-serving nodes} (see \cref{s:stateless-cryptocurrency:proof-serving-nodes}).

\parhead{The (Authenticated) State.}
The state is a vector $\vect{v}=(v_i)_{i\in[0,n)}$ of size $n$ that maps user $i$ to $v_i=(\addr_i|\bal_i)\in \Zp$, where $\bal_i$ is her balance and $\addr_i$ is her \textit{address}, which we define later.
(We discuss including transaction counters for preventing replay attacks in \cref{s:discussion:txn-counters}.)
Importantly, since $p \approx 2^{256}$, the first 224 bits of $v_i$ are used for $\addr_i$ and the last 32 bits for $\bal_i$.
%This means that $H(\cdot)$ outputs only 224 bits.
The genesis block's state is the all zeros vector with digest $d_0$
(e.g., in our aSVC, $d_0=g^0$).
Initially, each user $i$ is \textit{unregistered} and starts with a proof $\pi_{i,0}$ that their $v_i = 0$.
% i.e., $\forall i, \vcverifypos(\vrk, d_0, 0, i,\pi_{i,0})=T$.

\parhead{``Full'' vs. ``Traditional'' Public Keys.}
User $i$'s address is computed as $\addr_i=H(\PK_i)$, where $\PK_i=(i,\upk_i,\tpk_i)$ is her \textit{full public key}\ifNotCameraReady\xspace and $H$ is a collision-resistant hash function\fi.
Here, $\tpk_i$ denotes a \textit{``traditional'' public key} for a digital signature scheme, with corresponding secret key $\tsk_i$ used to authorize user $i$'s transactions.
To avoid confusion, we will clearly refer to public keys as either ``full'' or ``traditional.''

\parhead{Registering via \inittxn Transactions.}
\inittxn transactions are used to \textit{register} new users and assign them a unique, ever-increasing number from $1$ to $n$.
For this, each block $t$ stores a \textit{count of users registered so far} $\cnt_t$.
To register, a user generates a \textit{traditional secret key} $\tsk$ with a corresponding \textit{traditional public key} $\tpk$.
Then, she broadcasts an \inittxn transaction:
$$\tx=[\inittxn, \tpk]$$
A miner working on block ${t+1}$ who receives $\tx$, proceeds as follows.
\begin{enumerate}
\item He sets $i=\cnt_{t+1}$ and increments the count $\cnt_{t+1}$ of registered users,
\item He updates the VC via $d_{t+1}=\vccommupdate(d_{t+1}, (\addr_i\vert 0), i, \upk_i)$,
\item He incorporates $\tx$ in block $t+1$ as $\tx' = [\inittxn, (i, \upk_i,\tpk_i)]=[\inittxn, \PK_i]$.
\end{enumerate}
The full public key with $\upk_i$ is included so other users can correctly update their VC when they process $\tx'$.
\ifNotCameraReady
(The index $i$ is not necessary, since it can be computed from the block's $\cnt_{t+1}$ and the number of \inittxn transactions processed in the block so far.)
\fi
Note that to compute $\addr_i=H(\PK_i)$, the miner needs to have the correct $\upk_i$ which requires $O(n)$ storage.
We discuss how to avoid this in \cref{s:stateless-cryptocurrency:dos-update-key}.

\parhead{Transferring Coins via \spendtxn Transactions.}
When transferring $v$ coins to user $j$, user $i$ (who has $v'\ge v$ coins) must first obtain $\PK_j=(j, \upk_j,\tpk_j)$.
This is similar to existing cryptocurrencies, except the (full) public key is now slightly larger.
Then, user $i$ broadcasts a \spendtxn transaction, signed with her $\tsk_i$:
$$\tx=[\spendtxn, t, \PK_i, j, \upk_j, v,\pi_{i,t}, v']$$
% Note that miners will use $\tpk_i$ from $\PK_i$ to verify user $i$'s signature on this transaction, but they do not need user $j$'s $\tpk_j$ for anything.
% Don't they need \tpk_j to compute addr_j and update the digest with j's new balance?
% Nope, because only the value 'v' needs to be accumulated in the digest, while leaving addr_j intact.

A miner working on block ${t+1}$ processes this \spendtxn transaction as follows:
\begin{enumerate}
\item He checks that $v\le v'$ and verifies the proof $\pi_{i,t}$ that user $i$ has $v'$ coins via $\vcverifypos(\vrk, d_t, (\addr_i|v'), i, \pi_{i,t})$.
(If the miner receives another transaction from user $i$, it needs to carefully account for $i$'s new $v'-v$ balance.)
\item He updates $i$'s balance in block $t+1$ with $d_{t+1}=\vccommupdate(d_{t+1}, -v, i$, $\upk_i)$,
which only sets the lower order bits of $v_i$ corresponding to $\bal_i$, without touching the higher order bits for $\addr_i$.
\item He does the same for $j$ with $d_{t+1}=\vccommupdate(d_{t+1}$, $+v, j, \upk_j)$.
\end{enumerate}
% To verify a batch of txns from $i$, just verify all the proofs first, then update commitment!

\parhead{Validating Blocks.}
Suppose a miner receives a new block $t+1$ with digest $d_{t+1}$ that has $b$ \spendtxn transactions:
$$\tx=[\spendtxn, t, \PK_{i}, {j}, \upk_{j}, v, \pi_{i,t}, v']$$
To validate this block, the miner (who has $d_t$) proceeds in three steps (\inittxn transactions can be handled analogously):

\paragraph{Step 1: Check Balances.}
First, for each $\tx$, he checks that $v \le v'$ and that user $i$ has balance $v'$ via $\vcverifypos(\vrk, d_t$, $(\addr_{i}|v'), i, \pi_{i,t})=T$.
Since the sending user $i$ might have multiple transactions in the block, the miner has to carefully keep track of each sending user's balance to ensure it never goes below zero.

\paragraph{Step 2: Check Digest.}
Second, he checks $d_{t+1}$ has been computed correctly from $d_t$ and from the new transactions in block $t+1$.
Specifically, he sets $d'=d_t$ and for each $\tx$, he computes $d' = \vccommupdate(d', -v, i, \upk_{i})$ and $d'=\vccommupdate(d',+v$, $j,\upk_{j})$.
Then, he checks that $d'=d_{t+1}$.

\paragraph{Step 3: Update Proofs, If Any.}
If the miner lost the race to build block $t+1$, he can start mining block $t+2$ by ``moving over'' the \spendtxn transactions from his unmined block.
For this, he updates all proofs in those \spendtxn transactions, so they are valid against the new digest $d_{t+1}$.
%However, some of those transactions might now be invalid because the sending users' balance is too low.
% For example, this can happen if that user sent another transaction that got included in block $t+1$ and consumed too much of her balance.
Similarly, the miner must also ``move over'' all \inittxn transactions, since block $t+1$ might have registered new users.

\parhead{User Proof Synchronization.}
Consider a user $i$ who has processed the ledger up to time $t$ and has digest $d_t$ and proof $\pi_{i,t}$.
Eventually, she receives a new block $t+1$ with digest $d_{t+1}$ and needs to update her proof so it verifies against $d_{t+1}$.
Initially, she sets $\pi_{i,t+1} = \pi_{i,t}$.
For each $[\inittxn,\PK_j]$ transaction, she updates her proof $\pi_{i,t+1} = \vcproofupdate(\pi_{i,t+1}, (H(\PK_j)|0), i,j, \upk_i, \upk_j)$.
For each $[\spendtxn, t, \PK_j, k, \upk_k, v,\pi_{j,t}, v']$, she updates her proof twice: $\pi_{i,t+1} =\vcproofupdate(\pi_{i,t+1}, -v, i,j, \upk_i,\upk_j)$ and $\pi_{i,t+1} =\vcproofupdate(\pi_{i,t+1}$, $+v, i, k, \upk_i,\upk_k)$.
We stress that users can safely be offline and miss new blocks.
Eventually, when a user comes back online, she downloads the missed blocks, updates her proof and is ready to transact.

\subsection{Efficient Stateless Cryptocurrencies from aSVCs}
In this subsection, we explain how replacing the Edrax VC with our aSVC from \cref{s:asvc:from-kzg} results in a more efficient stateless cryptocurrency (see \cref{t:stateless-comparison}).
Then, we address a denial of service attack on user registrations in Edrax.

\subsubsection{Smaller, Faster, Aggregatable Proofs}
Our aSVC enables miners to aggregate all $b$ proofs in a block of $b$ transactions into a single, constant-sized proof.
This drastically reduces Edrax's per-block proof overhead from $O(b\log{n})$ group elements to just one group element.
Unfortunately, the $b$ update keys cannot be aggregated, but we still reduce their overhead from $O(b\log{n})$ to $b$ group elements per block (see \cref{s:stateless-cryptocurrency:smaller-update-keys}).
% (for users and miners to update their proofs and/or digest)
Our smaller proofs are also faster to update, taking $O(1)$ time rather than $O(\log{n})$.
While verifying an aggregated proof in our aSVC is $O(b\log^2{b})$ time, which is asymptotically slower than the $O(b)$ time for verifying $b$ individual ones, it is still \textit{concretely} faster as it only requires two, rather than $O(b)$, cryptographic pairings.
This makes validating new blocks much faster in practice.

\subsubsection{Reducing Miner Storage Using Verifiable Update Keys}
\label{s:stateless-cryptocurrency:dos-update-key}

We stress that miners must validate update keys before using them to update a digest.
Otherwise, they risk corrupting that digest, which results in a denial of service.
Edrax miners sidestep this problem by simply storing all $O(n)$ update keys.
Alternatively, Edrax proposes outsourcing update keys to an untrusted third party via a static Merkle tree.
Unfortunately, this would either require interaction \textit{during block proposal and block validation} or would double the update key size.
\ifNotCameraReady
For example, miners would need to fetch the correct update key and/or its Merkle proof to process a \spendtxn transaction.
\fi
Our implicitly-verifiable update keys avoid these pitfalls, since miners can directly verify the update keys in a \spendtxn transaction via \vcverifyupk.
Furthermore, for \inittxn transactions, miners can fetch (in the background) a running window of the update keys needed for the next $k$ registrations.
By carefully upper-bounding the number of registrations expected in the near future, we can avoid interaction during the block proposal.
This background fetching could be implemented in Edrax too, either with a small overhead via Merkle proofs or by making their update keys verifiable (which seems possible).

\subsubsection{Smaller Update Keys}
\label{s:stateless-cryptocurrency:smaller-update-keys}
Although, in our aSVC, $\upk_i$ contains $a_i=g^{A(\tau)/(X-\omega^i)}$ and $u_i=g^{\frac{\lagr_i(\tau)-1}{X-\omega^i}}$, miners only need to include $a_i$ in the block.
This is because of two reasons.
First, user $i$ already has $u_i$ to update her own proof after changes to her own balance.
Second, no other user $j\ne i$ will need $u_i$ to update her proof $\pi_j$.
However, as hinted in \cref{s:stateless-cryptocurrency:edrax}, miners actually need $u_i$ when only a subset of $i$'s pending transactions get included in block $t$.
In this case, the excluded transactions must have their proofs updated using $u_i$ so they can be included in block $t+1$.
Fortunately, this is not a problem, since miners always receive $u_i$ with user $i$'s transactions.
The key observation is that they do not have to include $u_i$ in the mined block, since users do not need it.

\subsubsection{Addressing DoS Attacks on User Registrations.}
\label{s:stateless-cryptocurrency:dos-registration}

Unfortunately, the registration process based on \inittxn transactions is susceptible to Denial of Service (DoS) attacks:
an attacker can simply send a large number of \inittxn transactions and quickly exhaust the free space in the vector $\vect{v}$.
There are several ways to address this.
First, one can use an aSVC from hidden-order groups, which supports an unbounded number of elements~\cite{CFG+20}.
However, that would negatively impact performance.
Second, as future work, one could develop and use unbounded, authenticated dictionaries \textit{with scalable updates}.
Third, one could simply use multiple bounded aSVCs together with cross-commitment proof aggregation, which our aSVC supports~\cite{GRWZ20}.
Lastly, one can add a cost to user registrations via a new \initspendtxn transaction that registers a user $j$ by having user $i$ send her some coins:
$$[\initspendtxn, t, \PK_i, \tpk,v,\pi_{i,t}, v'],\ \text{where}\ 0 < v \le v'$$
%User $i$ would sign this \initspendtxn transaction using her $\tpk_i$, similar to a \spendtxn transaction.
Miners processing this transaction would first register a new user $j$ with traditional public key $\tpk$ and then transfer her $v$ coins.
We stress that this is how existing cryptocurrencies operate anyway: in order to join, one has to be transferred some coins from existing users.
Lastly, we can ensure that each \tpk is only registered once by including in each \inittxn/\initspendtxn transaction a non-membership proof for $\tpk$ in a Merkle prefix tree of all TPKs.
We leave a careful exploration of this to future work.

Finally, miners (and only miners) will be allowed to create a \textit{single} $[\inittxn, \PK_i]$ transaction per block to register themselves.
This has the advantage of letting new miners join, without ``permission'' from other miners or users, while severely limiting DoS attacks, since malicious miners can only register a new user per block.
Furthermore, transaction fees and/or additional proof-of-work can also severely limit the frequency of \initspendtxn transactions.

\subsubsection{Minting Coins and Transaction Fees.}
Support for minting new coins can be added with a new \minttxn transaction type:
$$\tx = [\minttxn, i, \upk_i, v]$$
Here, $i$ is the miner's user account and $v$ is the amount of newly minted coins.
(Note that miners must register as users using $\inittxn$ transactions if they are to receive block rewards.)
%When this \minttxn transaction is processed by other users or miners, they update their digest $d_t$ using $\vccommupdate(d_t, +v, i, \upk_i)$.
%(In addition, users also update their proofs.)
To support transaction fees, we can extend the \spendtxn transaction format to include a fee, which is then added to the miner's block reward specified in the \minttxn transaction.

\subsection{Discussion}

\subsubsection{Making Room for Transaction Counters}
\label{s:discussion:txn-counters}
As mentioned in \cref{s:prelim:stateless-cryptocurrency}, to prevent transaction replay attacks, account-based stateless cryptocurrencies such as Edrax should actually map a user $i$ to $v_i = (\addr_i\vert c_i\vert\bal_i)$, where $c_i$ is her \textit{transaction counter}.
This change is trivial, but does leave less space in $v_i$ for $\addr_i$, depending on how many bits are needed for $c_i$ and $\bal_i$.
(Recall that $v_i\in\Zp$ typically has $\approx$ 256 bits.)
To address this, we propose using one aSVC for mapping $i$ to $\addr_i$ and another aSVC for mapping $i$ to $(c_i\vert\bal_i)$.
Our key observation is that if the two aSVCs use different $n$-SDH parameters (e.g., $(g^{\tau^i})$'s and $(h^{\tau^i})$'s, such that $\log_g{h}$ is unknown), then we could aggregate commitments, proofs and update keys so as to introduce zero computational and communication overhead in our stateless cryptocurrency.
Security of this scheme could be argued similar to security of perfectly hiding KZG commitments~\cite{KZG10a}, which commit to $\phi(X)$ as $g^{\phi(\tau)}h^{r(\tau)}$ in an analogous fashion.
We leave investigating the details of this scheme to future work.

\subsubsection{Overhead of Synchronizing Proofs}
\label{s:stateless-cryptocurrency:proof-serving-nodes}
In a stateless cryptocurrency, users need to keep their proofs updated w.r.t. the latest block.
For example, in our scheme, each user spends $O(b\cdot \Delta{t})$ time updating her proof, if there are $\Delta{t}$ new blocks of $b$ transactions each.
Fortunately, when the underlying VC scheme supports precomputing all $n$ proofs fast~\cite{Tomescu20}, this overhead can be shifted to untrusted third parties called \textit{proof-serving nodes}~\cite{CPZ18}.
Specifically, a proof-serving node would have access to the proving key \prk and periodically compute all proofs for all $n$ users.
Then, any user with an out-of-sync proof could ask a node for their proof and then manually update it, should it be slightly out of date with the latest block.
Proof-serving nodes save users a significant amount of proof update work, which is important for users running on constrained devices such as mobile phones.

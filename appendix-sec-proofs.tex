\section{Security Proofs}

\subsection{KZG Batch Opening Binding (Re)definition}
\label{s:kzg-batch-def}

We strengthen the \textit{batch opening binding definition} of KZG~\cite[Sec. 3.4, pg. 9]{KZG10a} and prove KZG still satisfies it.

\begin{definition}[Batch Opening Binding]
    \label{def:kzg:batch-opening-binding}
    $\forall$ adversaries $\Adv$ running in time $\poly(\lambda)$:
    \begin{align}
    \Pr \left[ \begin{array}{c}
    \mathsf{pp} \leftarrow \mathsf{KZG.Setup}(1^\lambda, n), \\
    c, I, J, v_I(X), v_J(X), \pi_I, \pi_J \leftarrow \Adv(\mathsf{pp}, 1^\lambda) : \\
    \mathsf{KZG.VerifyEvalBatch}(\mathsf{pp}, c, I, \pi_I, v_I(X)) = T\ \wedge \\
    \mathsf{KZG.VerifyEvalBatch}(\mathsf{pp}, c, J, \pi_J, v_J(X)) = T\ \wedge \\
    \exists k \in I\cap J,\ \text{such that}\ v_I(k) \ne v_J(k)
    \end{array} \right] \le \mathsf{negl}(\lambda)
    \end{align}
\end{definition}

Suppose an adversary breaks the definition.
Let $A_I(X)= \prod_{i\in I} (X - i)$.
Then, the following holds:
\begin{align}
e(c,g) &= e(\pi_I, g^{A_I(\tau)}) e(g^{v_I(\tau)},g)\\
e(c,g) &= e(\pi_J, g^{A_J(\tau)}) e(g^{v_J(\tau)},g)
\end{align}
Divide the top equation by the bottom one to get:
\begin{align}
\mathbf{1}_T &= \frac{e(g^{v_I(\tau)},g)}{e(g^{v_J(\tau)},g)} \frac{e(\pi_I, g^{A_I(\tau)})}{e(\pi_J, g^{A_J(\tau)})}\Leftrightarrow\\
\mathbf{1}_T &= e(g^{v_I(\tau)-v_J(\tau)},g) \frac{e(\pi_I, g^{A_I(\tau)})}{e(\pi_J, g^{A_J(\tau)})}\Leftrightarrow\\
e(g^{v_J(\tau)-v_I(\tau)},g) &= \frac{e(\pi_I, g^{A_I(\tau)})}{e(\pi_J, g^{A_J(\tau)})}
\end{align}
Let $v_k = v_I(k)$ and $v_k' = v_J(k)$.
We can rewrite $v_I(X)$ using the polynomial remainder theorem as $v_I(X)=q_I(X)(X-k) + v_k$.
Similarly, $v_J(X)=q_J(X)(X-k) + v_k'$.
\begin{align}
e(g^{q_J(\tau)(\tau - k) + v_k'-q_I(\tau)(\tau -k)-v_k},g) &= \frac{e(\pi_I, g^{A_I(\tau)})}{e(\pi_J, g^{A_J(\tau)})}\Leftrightarrow\\
e(g^{(\tau - k)(q_J(\tau) -q_I(\tau)) + v_k'-v_k},g) &= \frac{e(\pi_I, g^{A_I(\tau)})}{e(\pi_J, g^{A_J(\tau)})}\Leftrightarrow\\
e(g^{(\tau - k)(q_J(\tau) -q_I(\tau))}, g) e(g^{v_k'-v_k},g) &= \frac{e(\pi_I, g^{A_I(\tau)})}{e(\pi_J, g^{A_J(\tau)})}\Leftrightarrow\\
e(g^{q_J(\tau) -q_I(\tau)}, g)^{\tau - k} e(g,g)^{v_k'-v_k} &= \frac{e(\pi_I, g^{A_I(\tau)})}{e(\pi_J, g^{A_J(\tau)})}
\end{align}
Factor out $(X-k)$ in $A_I(X)$ to get $A_I(X)=a_I(X)(\tau - k)$.
Similarly, $A_J(X)=a_J(X)(\tau - k)$.
\begin{align}
e(g^{q_J(\tau) -q_I(\tau)}, g)^{\tau - k} e(g,g)^{v_k'-v_k} &= \left(\frac{e(\pi_I, g^{a_I(\tau)})}{e(\pi_J, g^{a_J(\tau)})}\right)^{\tau -k}\Leftrightarrow\\
e(g^{q_J(\tau) -q_I(\tau)}, g) e(g,g)^\frac{v_k'-v_k}{\tau - k} &= \frac{e(\pi_I, g^{a_I(\tau)})}{e(\pi_J, g^{a_J(\tau)})}\Leftrightarrow\\
e(g,g)^\frac{v_k'-v_k}{\tau - k} &= \frac{e(\pi_I, g^{a_I(\tau)})}{e(\pi_J, g^{a_J(\tau)}) e(g^{q_J(\tau) - q_I(\tau)}, g)}\Leftrightarrow\\
e(g,g)^\frac{1}{\tau - k} &= \left(\frac{e(\pi_I, g^{a_I(\tau)})}{e(\pi_J, g^{a_J(\tau)}) e(g^{q_J(\tau) - q_I(\tau)}, g)}\right)^{\frac{1}{v_k'-v_k}}
\end{align}

Since the commitments to $a_I(X),a_J(X),q_I(X),q_J(X)$ can be easily reconstructed from $v_I(X),v_J(X), I$ and $J$, and since $v_k'\ne v_k$, this constitutes a direct break of $n$-SBDH.

\subsection{Update Key Uniqueness}
\label{s:update-key-uniqueness-proof}

We prove that our aSVC scheme from \cref{s:svc} has \textit{Update Key Uniqueness} as defined in \cref{def:vc:update-key-uniquness}.
Let $a$ be the commitment to $A(X)=X^n - 1$ from the verification key \vrk.
Suppose an adversary outputs two update keys $\upk_i = (a_i, u_i)$ and $\upk_i'=(a_i',u_i')$ at position $i$ that both pass \vcverifyupk but $\upk_i\ne \upk_i'$.
Then, it must be the case that either $a_i \ne a_i'$ or that $u_i \ne u_i'$.

\parhead{$a_i\ne a_i'$ Case:}
Since both update keys pass verification, the following pairing equations hold:
\begin{align}
e(a_i, g^\tau / g^{\omega^i}) &= e(a, g)\\
e(a_i', g^\tau / g^{\omega^i}) &= e(a, g)
\end{align}

Thus, it follows that:
\begin{align}
e(a_i, g^\tau / g^{\omega^i}) &= e(a_i', g^\tau / g^{\omega^i})\Leftrightarrow\\
e(a_i, g) &= e(a_i', g)\Leftrightarrow\\
a_i = a_i'
\end{align}

Contradiction.

\parhead{$u_i\ne u_i'$ Case:}
Let $A'(X)$ denote the derivative of $A(X)=X^n - 1$.
Let $\ell_i=a_i^{1/A'(\omega^i)}=g^{\lagr_i(\tau)}$.

Since both update keys pass verification, the following pairing equations hold:
\begin{align}
e(\ell_i/g^1, g) &= e(u_i,g^\tau/g^{\omega^i})\\
e(\ell_i/g^1, g) &= e(u_i',g^\tau/g^{\omega^i})
\end{align}

Thus, it follows that:
\begin{align}
e(u_i,g^\tau/g^{\omega^i}) &= e(u_i',g^\tau/g^{\omega^i})\\
e(u_i, g) &= e(u_i', g)\Leftrightarrow\\
u_i = u_i'
\end{align}

Contradiction.
\section{Complexity of VCs Based on Hidden-order Groups}
\label{s:complexity:hog}

\begin{table*}[t]
    %\large
    %\small
    %\footnotesize
    \scriptsize
    \centering
    \caption{
        Asymptotic comparison of our aSVC with (aS)VCs based on hidden-order groups.
        $n$ is the vector size, $b$ is the subvector size, $\ell$ is the length in bits of vector elements, $N=n\ell$ and $\lambda$ is the security parameter.
        For schemes based on hidden-order groups, the complexities in the table are \textit{asymptotic} in terms group operations rather than exponentiations.
        This gives a better sense of performance, since exponents cannot be ``reduced'' in hidden-order groups as they can in known-order groups.
        We try to account for field operations (of size ${2\lambda}$ bits), but quantifying them precisely in these schemes can be very cumbersome.
        Also, since field operations are much faster, they can be mostly ignored.
        For our aSVC scheme, we give the same complexities in terms of group \textit{exponentiations}, pairings and field operations\ifNotCameraReady\xspace(see \cref{s:complexity-lagrange-asvc} for details)\fi.
        Because of this, the reader must be careful when comparing our scheme with the other schemes in this table: a group exponentiation in our scheme is roughly equivalent to $O(\lambda)$ group operations in the hidden-order group schemes.
        (*Updating the commitment in CFG$_\ell^1$ only works in a weaker security model where the commitment is not produced adversarially.)
    }
    \label{t:rsa-asvc-comparison-appendix} % must go after \caption{} for \cref{} to work
    \setlength{\tabcolsep}{.3em} % space between columns
    \begin{tabular}{lccccccccccccc}
        %\toprule
        {\makecell{(aS)VC scheme}}
        & \makecell{$\vert \prk\vert$}
        & \makecell{$\vert\vrk\vert$}
        & \makecell{$\vert \upk_i\vert$ or\\$\vert\uph_i\vert$}
        & \makecell{Com.}
        & \makecell{Com.\\upd.}
        & \makecell{$\vert\pi_i\vert$}
        & \makecell{Prove\\one\\$v_i$}
        & \makecell{Verify\\one\\$v_i$}
        & \makecell{Proof\\upd.}
        & \makecell{Prove\\subv.\\$(v_i)_{i\in I}$}
        & \makecell{Verify\\subv.\\$(v_i)_{i\in I}$}
        & \makecell{Aggr-\\egate}
        & \makecell{Prove\\each\\$(v_i)_{i\in[n]}$}
        \\
        \toprule

        BBF$_\ell$~\cite{BBF19} & $1$ & $1$ & $1$ & \bbfc & \rlgn & \rlgn\xspace bits & \bbfc & \bbvy & \nop & \bbfc & \bbvys & $b\llgn$         & $N\lg^2{N}$\\

        \cfgOne~\cite{CFG+20}   & $1$ & $1$ & 1   & \bbfc & $1^*$ & $1\ |\G_?|$       & \cfoo & \bbvy & 1    & \cfos & \bbvys & $b\lg{b}\lg{N}$  & $N\lg^2{N}$\\

        \cfgTwo~\cite{CFG+20}   & $1$ & $1$ & 1   & \cftc & 1     & $1\ |\G_?|$       & \cfto & $\ell$& 1    & \cfts & \cftvs & $\ell b\lg^2{b}$ & $N\lg{n}$\\

        \toprule

        \textbf{Our aSVC}       & $n$ & $b$ & $1$ & $n$   & $1$   & $1$               & $n$   & $1$   & $1$  & \btc  & \mlgsm & $b\lg^2{b}$      & \nlgn
    \end{tabular}
    %\toprule
\end{table*}

We give complexities of VCs based on hidden-order groups in \cref{t:rsa-asvc-comparison-appendix}.
These can be challenging to describe succinctly due to the many variable-length integer operations that arise.
In an effort to keep complexities simple without leaving out too much detail, we often measure (and even approximate) complexities in terms of operations in a finite field of size $2^{2\lambda}$ (e.g., additions, multiplications, computing \bezout coefficients, Shamir tricks), where $\lambda$ is our security parameter.
Another reason to do so is for fairness with VC schemes in known-order groups, which also use finite fields of size $2^{2\lambda}$.
Otherwise, a $2\lambda$-bit multiplication would be counted as $O(\lambda \log\lambda)$ in schemes such as BBF$_\ell$~\cite{BBF18}\footnote{Assuming recent progress on multiplying $b$-bit integers in $O(b\log{b})$ time.} and as $O(1)$ time for schemes like KZG (see \cref{s:complexity-kzg}).

\parhead{The Shamir Trick.}
The ``Shamir Trick''~\cite{Shamir83,BBF18} can be used to compute an $e$th root of $g$ given an $e_1$th root and an $e_2$th root where $e=e_1 e_2$ and $\gcd(e_1,e_2)=1$.
The idea is to compute \bezout coefficients $a,b$ such that $a e_1 + b e_2 = 1$.
Then, $\left(g^\frac{1}{e_1}\right)^b \left(g^\frac{1}{e_2}\right)^a=g^\frac{b e_2}{e_1 e_2} g^\frac{a e_1}{e_1 e_2} = g^\frac{a e_1 + b e_2}{e_1 e_2} = g^\frac{1}{e_1 e_2}$.
Note that $|a|\approx|e_2|$ and $|b|\approx|e_1|$.

\subsection{Complexity of BBF$_\ell$~\cite{BBF19}}
\label{s:complexity-bbf}
In this scheme, we assume the vector $\vect{v}=[v_1, v_2, \dots, v_n]$ is indexed from 1 to $n$.

\parhead{Public Parameters.}
Let $\ell$ denote the size of vector elements in bits.
Let $n$ denote the number of vector elements.
Let $N=\ell n$.
Let $\Gho$ denote a hidden-order group and $g$ be a random group element in $\Gho$.
Let $H : [N] \rightarrow \mathsf{Primes}$ be a bijective function that on input $i$ outputs the $i$th prime number $p_i$.
(Note that $|p_N|=\lln$ bits.)
The \prk and \vrk consist of only $g$.
This scheme does not use ``fixed'' update keys compatible with our definitions.
Instead, this scheme uses \textit{``dynamic'' update hints}: the $i$th update hint w.r.t. a commitment $c$ is a VC proof for $v_i$ that verifies against $c$.
In this sense, similar to Merkle trees, this scheme is less suitable for account-based stateless cryptocurrencies~\cite{CPZ18}, since it requires user $i$ to fetch user $j$'s proof too, before sending her money.

\parhead{Commitment.}
An $\ell$-bit vector element $v_i$ can be written as a vector of $\ell$ bits $(v_{i,j})_{j\in [0,\ell-1]}$
Then, each bit $v_{i,j}$ is mapped to the unique prime $p_{(i-1)\cdot \ell + j}$.
Put differently, each $v_i$ is mapped to $\ell$ unique primes $(p_{(i-1)\cdot \ell + 0}, p_{(i-1)\cdot \ell + 1},\dots$, $p_{(i-1)\cdot \ell+ (\ell - 1)})$.
Then, for each $v_i$, take the product of all primes corresponding to non-zero bits as $P_i = \prod_{j\in[0,\ell-1]} {v_{i,j}}\cdot \left(p_{(i-1)\cdot \ell + j}\right)$.
Note that $|P_i| = O(\ell \lln)$.
A commitment to the vector $\vect{v}=(v_i)_{i\in [n]}$ will be an RSA accumulator over these $P_i$'s:
\begin{align}
c
  &=g^{\prod_{i\in [n]}\prod_{j\in[0,\ell-1]} v_{i,j}\cdot \left( p_{(i-1)\cdot \ell + j} \right)}\\
  &=g^{\prod_{i\in [n]} P_i}
\end{align}

The exponent of $c$ is a product of at most $\ell n$ primes, with the biggest prime having size $O(\lln)$.
Thus, computing the $O(1)$-sized commitment $c$ takes $O(\ell n \lln)$ group operations.
(Note that, for hidden-order groups, we are counting group operations rather than exponentiations.
This is to give a better sense of performance, which varies with the exponent size, since exponents cannot be ``reduced'' in hidden-order groups.)

Since updating commitments requires update hints, which are VC proofs, we must first discuss VC proofs.

\subsubsection{Proofs for a $v_i$}
\label{s:complexity-bbf:compute-one-proof}
A proof $\pi_i$ for $v_i$ must show two things:
\begin{enumerate}
\item That $P_i$ corresponding to all non-zero bits is accumulated in $c$.
\item That $Z_i= \prod_{j\in[0,\ell-1]} {(1-v_{i,j})}\cdot \left(p_{(i-1)\cdot \ell + j}\right)$ corresponding to all zero bits is \textit{not} accumulated in $c$.
(Note that $|Z_i|=|P_i|=O(\ell\lln)$.)
\end{enumerate}

\parhead{Proving ``One'' Bits are Accumulated.}
To prove $P_i$ is ``in'', an $O(1)$-sized RSA accumulator subset proof w.r.t. $c$ can be computed with $O(\ell n \lln)$ group operations (via \textit{A}.$\mathsf{NonMemWitCreate}^*$ in~\cite[Sec 4.2, pg. 15]{BBF18}):
\begin{align}
\label{eq:bbf:pi_i1}
\pi_i^{[1]} &=g^{\prod_{j\in [n],j \ne i} P_j} = c^{1/P_i}
\end{align}

To speed up the verification of this (part of) the proof, a constant-sized \textit{proof of exponentiation (PoE)}~\cite{BBF18} is computed in $O(\ell\lln)$ field and group operations.
We discuss this later in~\cref{s:complexity-bbf:verify-one-proof}.
% These correspond to dividing $P_i$, which is $O(\ell \log{(\ell n)})$-sized by a random $O(\lambda)$-bit number in the PoE proof computation & verification (see~\cite[Sec 4.2.1, pg. 48]{Tomescu20}).
% We have to say it takes at least $\ell\log{\ell n}$ time, which is the size of $|P_i|$.

\parhead{Proving ``Zero'' Bits are Accumulated.}
To prove $Z_i$ is ``out'', an $O(\ell \lln)$-sized disjointness proof $\pi_i^{[0]}$ can be computed w.r.t. $c$ (via \textit{A}.$\mathsf{NonMemWitCreate}$ in~\cite[Sec 4.1, pg. 14]{BBF18}).
First, $Z_i$ must be computed, but we assume this can be done in $O(\ell\lln)$ field operations.
Second, \bezout coefficients are computed such that $\alpha \prod_{i\in n} P_i + \beta Z_i =1$.
Then, the disjointness proof is $\pi_i^{[0]}=(g^{\beta},\alpha)$.
Since $|\alpha| \approx |Z_i|$, the proof is $O(\ell \lln)$-sized.
Although this disjointness proof can be made $O(1)$-sized via \textit{proofs of knowledge of exponent (PoKE)} proofs, this seems to break the ability to aggregate VC proofs in BBF$_\ell$~\cite[Sec 5.2, pg. 20]{BBF18}.
However, the prover can still include two constant-sized  PoE proofs for $(g^{\beta})^{Z_i}$ and for $c^{\alpha}$ to make the verifier's job easier, which costs him only $O(\ell \lln)$ field and group operations.

To analyze the time complexity of computing $\pi_i^{[0]}$, recall that:
\begin{enumerate}
    \item The asymptotic complexity of computing \bezout coefficients on $b$-bit numbers is $O(b\log^2{b})$ time.
    \item $b=\left|\prod_{i\in n} P_i\right|=O(n\ell \lln)$.
\end{enumerate}
As a result, the \bezout coefficients take $O((n\ell\lln) \log^2{(n \ell\lln)=O(n\ell\lln (\log{n\ell} + \log{\lln})^2)}={}$\linebreak[4]$O(n\ell\log^3(\ell n))$ time.
However, since these are bit operations, we will count them as $O(n\ell \lln)$ field operations.
Furthermore, computing $g^{\beta}$, where $|\beta|\approx|\prod_{i\in[n]}P_i|=O(n\ell\lln)$ takes $O(n\ell\lln)$ group operations.

Overall, the time to compute $\pi_i$ is $O(\ell n\lln)=O(\ell n\log{n})$.

\subsubsection{Verifying a Proof for $v_i$}
\label{s:complexity-bbf:verify-one-proof}

To verify $\pi_i = (\pi_{i}^{[0]}, \pi_i^{[1]})$, the verifier proceeds as follows.
First, he computes $P_i$ in $O(\ell\lln)$ field operations.
Second, he checks that $\left(\pi_i^{[1]}\right)^{P_i} = c$ via the PoE proof in $\pi_i^{[1]}$ using $O(\lambda)$ group operations and $O(\ell\log{n})$ field operations.
Third, he parses $(g^{\beta}, \alpha)$ from $\pi_i^{[0]}$ and checks if $(g^{\beta})^{Z_i} c^\alpha=g$.
Since the prover included PoE proofs, this can be verified with $O(\lambda)$ group operations and $O(\ell\lln)$ field operations.

\subsubsection{Updates}
\label{s:complexity-bbf:proof-updates}

\parhead{Updating Commitments.}
Suppose $v_i$ changes to $v_i'$.
For message bits that are changed from $0$ to $1$, updating the commitment $c$ involves ``accumulating'' new primes associated with those bits in $c$.
For message bits that are changed from $1$ to $0$, updating $c$ involves removing the primes associated with those bits from $c$.
Recall that, in BBF$_\ell$, the $i$th update hint $\uph_i$ is actually the VC proof $\pi_i$ for $v_i$ w.r.t. $c$.
Also recall that $\pi_i^{[1]}=c^{1/P_i}$ from $\pi_i$ is exactly the commitment $c$ without any of the primes associated with $v_i$.
Thus, to update the commitment, we can compute $P_i'=\prod_{j\in [0,\ell-1]} v'_{i,j} p_{(i-1)\cdot\ell + j}$ in $O(\ell)$ field operations and set $c' = \left(\pi_i^{[1]}\right)^{P_i'}$ in $O(\ell \lln)$ group operations.

To process several updates for $b$ updated elements $(v_i)_{i\in I}$ with $\uph_i$'s that all verify w.r.t. $c$, we have to take an additional step.
First, we use $O(b)$ Shamir tricks on the $\pi_i^{[1]}$'s from $\uph_i$ to obtain the commitment $c^{1/\prod_{i\in I} P_i}$, which no longer accumulates any primes associated with the old elements $(v_i)_{i\in I}$.
Then, we can add back the new primes $P_i'$ associated with the new elements $(v'_i)_{i\in I}$ in $O(b\ell \lln)$ group operations.
We assume the $O(b)$ Shamir tricks can be done in $O(b)$ field operations.

\parhead{Updating Proofs.}
Proof updates are not discussed in~\cite{BBF19}, but seem possible.
We leave it to future work to describe them and their complexity.

\subsubsection{Subvector Proofs for $(v_i)_{i\in I}$}
Recall that a normal VC proof for $v_i$ reasons about which primes associated with $v_i$ are (not) accumulated in $c$.
A subvector proof will do the same, except it will reason about primes associated with all $(v_i)_{i\in I}$.
Thus, instead of reasoning about two $O(\ell \lln)$-sized $P_i$ and $Z_i$, it will reason about two $O(b\ell\lln)$-sized $\prod_{i\in I} P_i$ and $\prod_{i\in I} Z_i$.
Specifically, an $I$-subvector proof consists of:
\begin{align}
\pi_I^{[1]} &=g^{\prod_{j\in [n]\setminus I} P_j} = c^{1/\prod_{i\in I} P_i}\\
\pi_I^{[0]} &= \left(g^\beta, \alpha\right)\ \text{such that}\ (g^\beta)^{\prod_{i\in I} Z_i} c^\alpha = g
\end{align}

Let us analyze the proving time and the proof size.
First, computing $\pi_I^{[1]}$ can be done in $O(\ell (n-b) \lln)$ group operations, which is slightly faster than the $O(\ell n\lln)$ time for computing $\pi_i^{[1]}$ in an individual proof for $v_i$ (see \cref{eq:bbf:pi_i1}).
Second, computing the PoE for $\left(\pi_I^{[1]}\right)^{\prod_{i\in I} P_i} = c$ can be done in  $O(b\ell\lln)$ field and group operations.
Third, computing $\pi_I^{[0]}$ maintains the same asymptotic complexity, since it is dominated by the time to compute $g^\beta$, which remains just as expensive.
However, $\pi_I^{[0]}$'s size would increase to $O(b\ell\lln)$, since the \bezout coefficient $\alpha$ will be roughly of size $|\prod_{i\in I} Z_i|$.
Fortunately, the prover can avoid this by giving $c^\alpha$ rather than $\alpha$ along with a PoKE proof (i.e., one group element and one $2\lambda$-bit integer), while maintaining the same asymptotic complexity.
As before, the prover also gives a PoE proof for $\left(g^{\beta}\right)^{Z_i}$ to speed up the verifier's job.

Because of the PoE proof, verification of $\pi_I^{[1]}$ only requires $O(\lambda)$ group operations as before, but the number of field operations increases to $O(b\ell\lln)$.
Similarly, the PoKE proof will speed up verification of $\pi_I^{[0]}$ to $O(\lambda)$ group operations, but the $O(b\ell\lln)$ field operations remain for verifying the PoE proof for $\left(g^{\beta}\right)^{Z_i}$.

\subsubsection{Aggregating Proofs}
Since aggregating RSA membership and non-membership witnesses is possible~\cite{BBF18}, and BBF$_\ell$ VC proofs consist of one RSA membership (subset) proof and one non-membership (disjointness) proof, it follows that aggregating proofs is possible.
We leave it to future work to analyze the complexity of aggregation, which has to be at least $\Omega(b\ell\lln)$ since it must read all $b$ VC proofs as input, which are each $O(\ell\lln)$-sized.

\subsubsection{Precomputing All Proofs}
Computing all membership and non-membership witnesses for an RSA accumulators over $N$ elements is possible in $O(N\log{N})$ exponentiations~\cite{BBF18,SSY01}.
Since for BBF$_\ell$ we have $N=\ell n$ and an exponentiation costs $O(\lln)$ group operations, this would take $O(\ell n \log^2{(\ell n)})$ group operations.
We are ignoring (1) the overhead of aggregating membership and non-membership witnesses and (2) the overhead of computing PoE proofs, which we assume is dominated by the cost to compute the witnesses.

\subsection{Complexity of \cfgOne~\cite{CFG+20} and \cfgTwo~\cite{CF13,LM19,CFG+20}}
% In Sec 3.7, they say both universal and specialized CRS has length 1.

We refer the reader to~\cite[Table 1, pg. 35]{CFG+20} for most of these these complexities.

\parhead{Aggregating Proofs.}
For CFG$_{\ell}^1$, aggregating $b$ proofs into an $I$-subvector proof takes $O(b\log{b}\log{N})$ group operations~\cite[Sec 5.1, pg. 23]{CFG+20}.
For CFG$_{\ell}^2$, this takes $O(\ell b\log^2{b})$ group operations~\cite[Sec 5.2, pg. 32]{CFG+20}.

\parhead{Precomputing All Proofs.}
The paper gives a generic technique of using incremental (dis)aggregation to precompute auxiliary information for serving proofs fast.
This technique can also be used to precompute all proofs fast in quasilinear time.
In CFG$_\ell^{1}$, we believe this will take $O(N\log^2{N})$ group operations, dominated by the complexity of computing all $N=\ell n$ RSA accumulator membership witnesses.
In CFG$_\ell^{2}$, we estimate this will take $O(\ell n\log{n})$ group operations (since disaggregating a proof of size $m$ into two proofs of size $m/2$ takes $O(\ell m)$ group operations).

\section{Complexities of Pairing-based VCs in \cref{t:asvc-comparison}}
\label{s:complexity:kog}

We survey each \textit{pairing-based} VC scheme from \cref{t:asvc-comparison} and explain its complexities.
In \cref{s:complexity:hog}, we do the same for VCs based on hidden-order groups.
Despite our best efforts to understand the complexities of each scheme, we recognize there could be better upper bounds for some of them.

\subsection{Complexities of CDH-based~\cite{LM19}}

This scheme was originally proposed by Catalano and Fiore~\cite{CF13} and extended by Lai and Malavolta to support subvector proofs~\cite{LM19}.

\parhead{Public Parameters.}
% |\prk| = 1+n+n/2 to be more exact
The proving key is $\prk=(h_{i,j})_{i,j\in [0,n)}$ and is $O(n^2)$ sized.
Here, $h_{i,j}=g^{z_i \cdot z_j}$ when $i\ne j$ and $h_{i,i} = h_i=g^{z_i}$, with each $z_i \in \Zp$ picked uniformly at random.
The verification key is $\vrk=(h_i)_{i\in [0,n)}$ and is $O(n)$-sized.
% |\vrk|: each player needs h_i to verify position i
% (unclear if you can include h_i with proof: what could I verify $h_i = h^{z_i}$ against?)
The $i$th update key is $\upk_i = (h_i, (h_{i,j})_{j\in [0,n)})$.
% |\upk_i|: player $i$ needs to include $h_{i,j}$ for all $j$ so any player $j$ can update her proof
Note that $h_{i,j} = h_j^{z_i} = h_i^{z_j}$.

\parhead{Commitment.}
A commitment is $c=\prod_{i\in [0,n)} h_i^{v_i}$ and can be computed with $O(n)$ exponentiations.
If any vector element $v_j$ changes to $v_j + \delta$, the commitment can be updated in $O(1)$ time using $h_j$ from $\upk_j$ as $c' = c \cdot (h_j)^{\delta}$.

\parhead{Proofs for a $v_i$.}
A proof for $v_i$ is:
\begin{align}
\pi_i = \prod_{j\in [0,n)\setminus\{i\}} h_{i,j}^{v_j}=\left(\prod_{j\in [0,n)\setminus\{i\}} h_{j}^{v_j}\right)^{z_i}
\end{align}
The proof is $O(1)$-sized and can be computed from the $h_{i,j}$'s in the \prk with $O(n)$ exponentiations.
It can be verified in $O(1)$ time using $h_i$ from the \vrk as:
\begin{align}
e(C/h_i^{v_i}, h_i) = e(\pi_i, g)
\end{align}
If any vector element $v_j,j\ne i$ changes to $v_j + \delta$, the proof $\pi_i$ can be updated in $O(1)$ time using $h_{i,j}$ from $\upk_j$ as $\pi_i' = \pi_i \cdot \left(h_{i,j}^{\delta}\right)$.
This new $\pi_i'$ will verify against the updated $c'$ commitment defined earlier.
If $v_i$ changes to $v_i + \delta$, the proof $\pi_i$ need not be updated.

\parhead{Subvector Proofs for $(v_i)_{i\in I}$}
A $O(1)$-sized subvector proof for $\vect{v}_I$ is:
\begin{align}
\pi_I = \prod_{i \in I}\prod_{j \in [0,n)\setminus I} h_{i,j}^{v_j}=\prod_{i \in I}\left(\prod_{j \in [0,n)\setminus I} h_{j}^{v_j}\right)^{z_i}=\prod_{i\in I} \pi_i^*
\end{align}
As intuition, note that the inner product $\pi_i^*=\left(\prod_{j \in [0,n)\setminus I} h_{j}^{v_j}\right)^{z_i}$ is very similar to a proof $\pi_i=\left(\prod_{j \in [0,n) \setminus \{i\}} h_{j}^{v_j}\right)^{z_i}$ for $v_i$.
Let $b=|I|$.
The proof can be computed from the $h_{i,j}$'s in the \prk with $O(b(n-b))$ exponentiations (because each $\pi_i^*$ can be computed in $O(n-b)$ exponentiations).
% Note: It might seem like things repeat, but the bases $h_{i,j}$ are different for each $i$
A subvector proof $\pi_I$ can be verified using $(h_i)_{i\in I}$ from \vrk by checking in $O(b)$ time if:
\begin{align}
e\left(c/\prod_{j\in I} h_j^{v_j}, \prod_{i\in I} h_i\right) &= e(\pi_I, g)\Leftrightarrow\\
e\left(\prod_{j\in [0,n)\setminus I} h_j^{v_j}, \prod_{i\in I} g^{z_i}\right) &= e\left(\prod_{i \in I}\prod_{j \in [0,n)\setminus I} h_{i,j}^{v_j}, g\right)\\
e\left(\prod_{j\in [0,n)\setminus I} h_j^{v_j}, g^{\sum_{i\in I} z_i}\right) &= e\left(\prod_{i \in I}\left(\prod_{j \in [0,n)\setminus I} h_{j}^{v_j}\right)^{z_i}, g\right)\\
e\left(\left(\prod_{j\in [0,n)\setminus I} h_j^{v_j}\right)^{\sum_{i\in I} z_i},g\right) &= e\left(\left(\prod_{j \in [0,n)\setminus I} h_{j}^{v_j}\right)^{\sum_{i \in I}z_i}, g\right)
\end{align}
% TODO: Aggregation seems possible in CDH-based CF13/LM19: Just tweak a $\pi_i$ into a $\pi_i^*$, by removing $h_{i,j}^v_j$ \forall j \in I\setminus\{i\}$. However, it requires $h_{i,j}$'s $\forall i,j\in I$, I think.
% TODO: Updating subvector proofs seems possible too.

\parhead{Aggregating Proofs and Precomputing All Proofs.}
Aggregating proofs is not discussed in~\cite{CF13,LM19}, but it seems possible.
Finally, precomputing all proofs efficiently is not discussed.
Naively, it can be done inefficiently in $O(n^2)$ time.

\subsection{Complexities of KZG~\cite{KZG10a}}
\label{s:complexity-kzg}
Kate, Zaverucha and Goldberg also discuss using their polynomial commitment scheme~\cite{KZG10a} to commit to a sequence of messages, thus implicitly obtaining a VC scheme.
Although they do not analyze its complexity in their paper, we do so here.

\parhead{Public Parameters.}
The proving key is $\prk=(g^{\tau^i})_{i\in [0,n-1]}$ and is $O(n)$ sized.
The verification key is $\vrk=(g,(g^{\tau^i})_{i\in b})$, where $b=|I|$ is the size of the largest subvector whose proof the verifier should be able to check, and is thus $O(b)$-sized.
There is no support for updating commitments and proofs using update keys, although our work shows this is possible (see \cref{s:asvc:from-kzg}).

\parhead{Commitment.}
A commitment is $c=g^{\phi(\tau)}$ where $\phi(X)=\sum_{i\in [0,n)} \lagr_i(X) v_i$ and can be computed with $O(n\log^2{n})$ field operations (see \cref{s:prelim:interpolation}) and $O(n)$ exponentiations.
Commitment updates are not discussed, but the scheme could be modified to support them (see \cref{s:asvc:from-kzg}).

\parhead{Proofs for a $v_i$.}
A proof for $v_i$ is:
\begin{align}
\pi_i = g^{\frac{\phi(\tau)-v_i}{\tau - i}} = g^{q_i(\tau)}
\end{align}
The proof is $O(1)$-sized and can be computed by dividing $\phi(X)$ by $(X-i)$ in $O(n)$ field operations, obtaining $q_i(X)$, and committing to $q_i(X)$ using the $g^{\tau^i}$'s in the \prk with $O(n)$ exponentiations.
The proof can be verified in $O(1)$ time using $g^\tau$ from the \vrk by computing two pairings:
\begin{align}
e(c/g^{v_i}, g) = e(\pi_i, g^{\tau}/g^{i})
\end{align}
Proof updates are not discussed, but the scheme could be modified to support them (see \cref{s:asvc:from-kzg}).

\parhead{Subvector Proofs for $(v_i)_{i\in I}$}
A $O(1)$-sized subvector proof for $\vect{v}_I$ is:
\begin{align}
\pi_I=g^\frac{\phi(\tau)-R_I(\tau)}{A_I(\tau)}=g^{q_I(\tau)}
\end{align}
Here, $R_I(X)$ of degree $\le b-1$ is interpolated in $O(b\log^2{b})$ field operations so that $R_I(i) = v_i,\forall i\in I$ (see \cref{s:prelim:interpolation}).
Also, $A_I(X)= \prod_{i\in I} (X-i)$ is computed in $O(b\log^2{b})$ field operations via a \textit{subproduct tree}~\cite{vG13ModernCh10}.
The proof leverages the fact that dividing $\phi(X)$ by $A_I(X)$ gives quotient $q_I(X)$ and remainder $R_I(X)$.
The quotient $q_I(X)$ can be obtained in $O(n\log{n})$ field operations via a DFT-based division~\cite{vG13ModernCh9}.
Given $g^{\tau^i}$'s from the \prk, committing to $q_I(X)$ takes $O(n-b)$ exponentiations, since $\deg(q_I)=\deg(\phi)-\deg(A_I)\le (n-1)-b$.
Thus, the overall subvector proving time is $O(n\log{n}+b\log^2{b})$.

To verify a subvector proof $\pi_I$, first, the verifier must recompute $R_I(X)$ and $A_I(X)$ in $O(b\log^2{b})$ field operations.
Then, the verifier uses $(g^{\tau^i})_{i\in b}$ from the \vrk to compute KZG commitments $g^{R_I(\tau)}$ and $g^{A_I(\tau)}$ in $O(b)$ exponentiations.
Finally, the verifier checks using two pairings if:
\begin{align}
e(c/g^{R_I(\tau)},g) &= e(\pi_I, g^{A_I(\tau)})
\end{align}
Thus, the overall subvector proof verification time is $O(b\log^2{b})$ time.

\parhead{Aggregating Proofs and Precomputing All Proofs.}
Aggregating proofs is not discussed, but the scheme can be modified to support them (see \cref{s:asvc:from-kzg:aggregating-proofs}).
Finally, precomputing all proofs efficiently is not discussed, but is possible (see \cref{s:asvc:from-kzg}).
Naively, it can be done inefficiently in $O(n^2)$ time.

\subsection{Complexity of CDHK~\cite{CDHK15}}
\label{s:complexity-kzg-lagr}
In this scheme, we assume the vector $\vect{v}=[v_1, v_2, \dots, v_n]$ is indexed from 1 to $n$.
This scheme is similar to a KZG-based VC, except (1) it is randomized, (2) it computes proofs in a slightly different way and (3) it \textit{willfully} prevents aggregation of proofs as a security feature.

\parhead{Public Parameters.}
% Note: Need g^{\tau^{n+1}} for accumulator $X\prod_{i\in [n]}(X-i)$
The proving key is $\prk=\left((g^{\tau^i})_{i\in [0,n+1]}, (g^{\lagr_i(\tau)})_{i\in[0,n]},g^{P(\tau)}\right)$, where $P(X)=X\cdot \prod_{i\in[n]} (X-i)$ and is $O(n)$ sized.
(Note that the Lagrange polynomials $\lagr_i(X)=\prod_{j\in[0,n],j\ne i} \frac{X-j}{i-j}$ are defined over the points $[0,n]$, not $[n]$.)
The verification key is $\vrk=(g,(g^{\lagr_i(\tau)})_{i\in [n]},(g^{\tau^i})_{i\in [0,b+1]})$, where $b=|I|$ is the size of the largest subvector whose proof the verifier should be able to check.
Unfortunately, the verification key is $O(n)$-sized.
There is no support for updating commitments and proofs using update keys, although adding it is possible via our techniques (see \cref{s:asvc:from-kzg}).
As a result, we treat this scheme as if it used an $O(n)$-sized update key $\upk_i=\left(g^{\lagr_i(\tau)}, \left(g^\frac{\lagr_j(\tau)}{\tau - i}\right)_{j\in[n]}\right)$

\parhead{Commitment.}
A commitment is $c=\prod_{i\in[n]} \left(g^{\lagr_i(\tau)}\right)^{v_i} \left(g^{P(\tau)}\right)^{r}= g^{\phi(\tau) + r\cdot P(\tau)}$ where $\phi(X)=\sum_{i\in [0,n]} \lagr_i(X) v_i$, with $v_0 = 0$.
To compute the commitment, $\phi(X)$ must first be interpolated using $O(n\log^2{n})$ field operations.
Then, $c$ can be computed with $O(n)$ exponentiations, given the Lagrange commitments and $g^{P(\tau)}$ from \prk.
Commitment updates are not discussed, but they can be trivially implemented by setting $\upk_i = g^{\lagr_i(\tau)}$ and having $c' = c \cdot \left(g^{\lagr_j(\tau)}\right)^{\delta}$ be the new commitment after a change $\delta$ to $v_j$.
We reflect this in \cref{t:asvc-comparison}.

\parhead{Proofs for a $v_i$.}
A proof for $v_i$ is:
\begin{align}
\pi_i = g^{\frac{\left(\phi(\tau)+r\cdot P(\tau)\right)-v_i\lagr_i(\tau)}{\tau - i}} = g^{q_i(\tau)}
\end{align}
The proof is $O(1)$-sized and can be computed by dividing $\phi(X)+r\cdot P(X)-v_i\lagr_i(X)$ by $(X-i)$ in $O(n)$ field operations, obtaining $q_i(X)$, and committing to $q_i(X)$ using the $g^{\tau^i}$'s in the \prk with $O(n)$ exponentiations.
The proof can be verified in $O(1)$ time using $g^{\lagr_i(\tau)}$ from the \vrk by computing two pairings:
\begin{align}
e\left(c/\left(g^{\lagr_i(\tau)}\right)^{v_i}, g\right) = e(\pi_i, g^{\tau}/g^{i})
\end{align}
Proof updates are not discussed, but the scheme could be modified to support them (see \cref{s:asvc:from-kzg}).

\parhead{Subvector Proofs for $(v_i)_{i\in I}$}
A $O(1)$-sized subvector proof for $\vect{v}_I$ is:
\begin{align}
\pi_I=g^\frac{\phi(\tau)+r\cdot P(\tau) - R_I(\tau)}{A_I(\tau)}=g^{q_I(\tau)}
\end{align}
Here, $R_I(X)$ is defined so that $R_I(i) = v_i,\forall i\in I$ \textbf{and} $R_I(i) = 0, \forall i \in [0,n]\setminus I$.
(In particular, this means $R_I(0) = 0$.)
Interpolating $R_I(X)$ takes $O(n\log^2{n})$ field operations.
Also, $A_I(X)= X\prod_{i\in I} (X-i)$ is computed in $O(b\log^2{b})$ field operations via a \textit{subproduct tree}~\cite{vG13ModernCh10}.
Given $g^{\tau^i}$'s from the \prk, committing to $q_I(X)$ takes $O(n-b)$ exponentiations (because $\deg(q_I) = \deg(\phi)-\deg(A_I)\le n-(b+1)$).
Thus, the overall subvector proving time is $O(n\log^2{n})$.

To verify a subvector proof $\pi_I$, first, the verifier recomputes the commitment to $g^{R_I(\tau)}=\sum_{i\in I} \left(g^{\lagr_i(\tau)}\right)^{v_i}$ using $O(b)$ exponentiations.
(Recall that $\lagr_i(X)$ is defined over $[0,n]$ and has its KZG commitment in the \vrk.)
Then, he computes $A_I(X)$ in $O(b\log^2{b})$ field operations using a \textit{subproduct tree}~\cite{vG13ModernCh10}.
Then, the verifier uses $(g^{\tau^i})_{i\in [0,b+1]}$ from the \vrk to compute a KZG commitment to $g^{A_I(\tau)}$ in $O(b)$ exponentiations.
Finally, the verifier checks if:
\begin{align}
e(c/g^{R_I(\tau)},g) &= e(\pi_I, g^{A_I(\tau)})
\end{align}
Thus, the overall subvector proof verification time is $O(b\log^2{b})$.

\parhead{Aggregating Proofs and Precomputing All Proofs.}
Aggregating proofs is \textit{willfully} prevented by this scheme, as a security feature.
Finally, precomputing all proofs efficiently is not discussed, but it can be done inefficiently in $O(n^2)$ time.
Importantly, because the proofs are slightly different from KZG, they are not amenable to known techniques for precomputing all $n$ proofs in $O(n\log{n})$ time~\cite{FK20}.
% TODO: Unclear if faster than $O(n^2)$ time OpenAll algorithm exists, given the proof verification using $-v\lagr_i(\tau)$.

\subsection{Complexities of CPZ~\cite{CPZ18}}
\label{s:complexity:cpz}

Since the Edrax paper clearly summarizes its performance, we refer the reader to~\cite[Table 1]{CPZ18}, with one exception discussed below.

\parhead{Aggregating Proofs and Precomputing All Proofs.}
Aggregating proofs is not discussed and it is unclear if the scheme can be modified to support it.
Precomputing all proofs efficiently is not discussed either to the best of our knowledge, but could be possible.
% The key idea is to notice that computing $n$ proofs separately actually repeats a lot of work.
% If we avoid re-doing previously-done work, all proofs can be computed in $O(n\log{n})$ time.

\subsection{Complexities of TCZ~\cite{TCZ+20,Tomescu20}}
\label{s:complexity-tcz}

In their paper on scaling threshold cryptosystems, Tomescu et al.~\cite{TCZ+20} present a technique for computing $n$ \textit{logarithmic-sized} evaluation proofs for a KZG committed polynomial of degree $t$ in $O(n\log{t})$ time.
Later on, Tomescu extends these results to obtain a full VC scheme~\cite[Sec  9.2]{Tomescu20}.

\parhead{Public Parameters.}
The proving key is $\prk=((g^{\tau^i})_{i\in [0,n-1]},(g^{\lagr_i(\tau)})_{i\in [0,n)})$ and is $O(n)$ sized.
Importantly, $n$ is assumed to be a power of two, and $\lagr_i(X)=\prod_{j\in [0,n),j\ne i} \frac{X-\omega^j}{\omega^i - \omega^j}$ where $\omega$ is a primitive $n$th root of unity~\cite{vG13ModernCh8}.
The verification key is $\vrk=(g,(g^{\tau^{2^i}})_{i\in [\floor{\log_2{(n-1)}}]}, (g^{\tau^i})_{i\in[b]})$, where $b=|I|$ is the size of the largest subvector whose proof the verifier should be able to check, and is thus $O(b)$-sized.
The $i$th update key $\upk_i$ is the \textit{authenticated multipoint evaluation tree (AMT)} of $\lagr_i(X)$ at all points $(\omega^i)_{i\in [0,n)}$ (see \cite[Sec III-B]{TCZ+20} and~\cite[Ch 9]{Tomescu20}).
This AMT will be $O(\log{n})$-sized, consisting of a single path of non-zero quotient commitments leading to the evaluation of $\lagr_i(\omega^i)$~\cite[Sec 9.2.2]{Tomescu20}, since all other evaluations $\lagr_i(\omega^j), j\ne i$ are zero.

\parhead{Commitment.}
A commitment is $c=g^{\phi(\tau)}$ where $\phi(\omega^i)=v_i,\forall i\in[0,n)$.
Note that $\phi(X)$ can be computed with $O(n\log{n})$ field operations via an inverse Discrete Fourier Transform (DFT)~\cite[Ch 30.2]{CLRS09}.
Then, computing $c$ requires $O(n)$ exponentiations.
Commitment updates remain the same as in the KZG-based scheme from \cref{s:complexity-kzg-lagr}: $c' = c\cdot \left(g^{\lagr_j(\tau)}\right)^{\delta}$, where $\delta$ is the change at position $j$ in the vector and the Lagrange polynomial commitment can be obtained from $\upk_j$.

\parhead{Proofs for a $v_i$.}
A proof for $v_i$ is:
\begin{align}
\pi_i = (g^{q_w(\tau)})_{w\in \left[1+\floor{\log{(n-1)}}\right]}
\end{align}
Here, each $q_w(X)$ is a quotient polynomial along the AMT tree path to $\phi(\omega^i)$.
The proof is $O(\log{n})$-sized and can be computed by ``repeatedly'' dividing $\phi(X)$ by accumulator polynomials of ever-decreasing sizes $n/2, \dots, 4,2,1$ in $T(n)=O(n\log{n}) + T(n/2) = O(n\log{n})$ field operations, and committing to each $q_w(X)$ using the $g^{\tau^i}$'s in the \prk with $T'(n) = O(n) + T'(n/2)=O(n)$ exponentiations.
(``Repeatedly'' dividing means we first divide $\phi(X)$ by a degree $n/2$ accumulator. Then, we take the remainder of this division and divide it by the degree $n/4$ accumulator. We then take this remainder and divide it by a degree $n/8$ accumulator. And so on. This ensures the remainder degrees always halve.)
% i.e., T(n) = n(\log{n} + 1/2\log{n/2} + 1/4 \log{n/4} + ...) \le n(\log{n} + 1/2 \log{n} + 1/4 \log{n} + ....) \le n(2*\log{n})
The proof can be verified in $O(\log{n})$ time using the $g^{\tau^{2^i}}$'s from the \vrk:
\begin{align}
e(c/g^{v_i}, g) = \prod_{w\in \left[1+\floor{\log{(n-1)}}\right]}e(g^{q_w(\tau)}, g^{a_w(\tau)})
\end{align}
Here, the $a_w(X)$'s denote the accumulator polynomials along the AMT path to $\phi(\omega^i)$, which are always of the form $X^{2^i} - c$ for some constant $c$ and some $i\in [0,\floor{\log{(n-1)}}]$.

\parhead{Proof Updates.}
If any vector element $v_j$ changes to $v_j + \delta$, the proof $\pi_i$ can be updated in $O(\log{n})$ time.
(It could be that $j=i$.)
The idea is to consider the quotient commitments $g^{q_{w}(\tau)}$ along $\pi_i$'s AMT path and the $g^{u_w(\tau)}$ commitments along $\upk_j$'s AMT path.
For all locations $w$ where the two paths intersect, the quotient commitments are combined in constant time as $g^{q'_w(\tau)} = g^{q_w(\tau)} \cdot \left(g^{u_w(\tau)}\right)^\delta.$
Since there are at most $O(\log{n})$ locations $w$ to intersect in, this takes $O(\log{n})$ exponentiations.
This new $\pi_i'$ with quotient commitments $g^{q'_w(\tau)}$ will verify against the updated $c'$ commitment defined earlier.

\parhead{Subvector Proofs for $(v_i)_{i\in I}$}
This scheme uses the same subvector proof as the original KZG-based scheme in \cref{s:complexity-kzg}.
Thus, the subvector proving time is $O(n\log{n}+b\log^2{b})$ and the verification time is $O(b\log^2{b})$ time.

\parhead{Aggregating Proofs and Precomputing All Proofs.}
Aggregating proofs is not discussed and it is unclear if the scheme can be modified to support it.
Precomputing all \textit{logarithmic-sized} proofs efficiently is possible via the AMT technique in $O(n\log{n})$ time.

\subsection{Complexity of Pointproofs~\cite{GRWZ20,LY10}}
\label{s:complexity:pointproofs}
Gorbunov et al.~\cite{GRWZ20} enhance the VC by Libert and Yung~\cite{LY10} with the ability to aggregate multiple VC proofs into a subvector proof.
Additionally, they also enable aggregation of subvector proofs across different vector commitments, which they show is useful for stateless smart contract validation in cryptocurrencies.
In this scheme, we assume the vector $\vect{v}=[v_1, v_2, \dots, v_n]$ is indexed from 1 to $n$.

\parhead{Public Parameters.}
Their scheme works over Type III pairings $e : \G_1 \times \G_2\rightarrow \G_T$.
Let $g_1,g_2,g_T$ be generators of $\G_1, \G_2$ and $\G_T$ respectively.
The proving key $\prk = (g_1, (g_1^{\alpha^i})_{i\in[1,2n]\setminus\{n+1\}},g_2,(g_2^{\alpha^i})_{i\in[1,n]}, g_T^{\alpha^{n+1}})$.
Note that $g_1^{\alpha^{n+1}}$ is ``missing'' from the proving key, which is essential for security.
The verification key $\vrk=((g_2^{\alpha^i})_{i\in[1,n]},g_T^{\alpha^{n+1}})$ is $O(n)$ -sized.
The $i$th update key is $\upk_i=g_1^{\alpha^i}$.
They only support updating commitments, but proofs could be made updatable at the cost of linear-sized update keys.

\parhead{Commitment.}
A commitment is $c=\prod_{i\in[n]} \left(g_1^{\alpha^i}\right)^{v_i}=g_1^{\sum_{i\in [n]} v_i\alpha^i}$ and can be computed with $O(n)$ exponentiations.
If any vector element $v_j$ changes to $v_j + \delta$, the commitment can be updated in $O(1)$ time as $c' = c \cdot (\upk_j)^{\delta} = c\cdot (g_1^{\alpha^j})^{\delta}$.

\parhead{Proofs for a $v_i$.}
A proof for $v_i$ is obtained by re-committing to $v$ so that $v_i$ ``lands'' at position $n+1$ (i.e., has coefficient $\alpha^{n+1}$) rather than position $i$ (i.e., has coefficient $\alpha^i$).
Furthermore, this commitment will \textbf{not} contain $v_i$: it cannot, since that would require having $g_1^{\alpha^{n+1}}$.
To get position $i$ to $n+1$, we must ``shift'' it (and every other position) by $(n + 1) - i$.
Thus, the proof is:
\begin{align}
\pi_i &= g_1^{\sum_{j\in[n]\setminus\{i\}} v_j \alpha^{j + (n+1) - i}}\\
      &= g_1^{\sum_{j\in[n]\setminus\{i\}} v_j \alpha^{j} \alpha^{(n+1) - i}}\\
      &= \left(g_1^{\sum_{j\in[n]\setminus\{i\}} v_j \alpha^{j}}\right)^{\alpha^{(n+1) - i}}\\
      &= \left(\frac{g_1^{\sum_{j\in[n]} v_j \alpha^{j}}}{g_1^{v_i \alpha^i}}\right)^{\alpha^{(n+1) - i}}\\
      &= (c / g_1^{v_i \alpha^i})^{\alpha^{(n+1) - i}}
\end{align}
The proof is constant-sized and can be computed with $O(n)$ exponentiations.
It can be verified in $O(1)$ time using $g_2^{\alpha^{(n+1) - i}}$ from \vrk:
\begin{align}
e(c, g_2^{\alpha^{(n+1)-i}})=e(\pi_i, g_2) \left(g_T^{\alpha^{n+1}}\right)^{v_i}
\end{align}
Updating the proof is not discussed but can be done in $O(1)$ time, if the update keys are tweaked to be linear rather than constant-sized.

\parhead{Subvector Proofs for $(v_i)_{i\in I}$}
An $O(1)$-sized subvector proof for $\vect{v}_I$ is just a random linear combination of all proofs $\pi_i,\forall i\in I$.
First, all $b$ proofs $\pi_i$ are computed in $O(bn)$ exponentiations as described above.
Second, for each $i\in I$, $t_i = H(c, I, (v_i)_{i\in I})$ is computed using a random oracle $H : \{0,1\}^*\rightarrow \Zp$.
Third, the subvector proof $\pi_I$ is computed as:
\begin{align}
\pi_I = \prod_{i\in I} \pi_i^{t_i}
\end{align}
If computed this way, a subvector proof would take $O(bn)$ exponentiations.
However, Gorbunov et al. observe that $\pi_I$ can be computed with an $O(n)$-sized multi-exponentiation on a subset of the $2n$ generators $(g_1^{\alpha^i})_{i\in[0,2n]\setminus \{n+1\}}$.
The exponents will be a combination of the messages and the $t_i$'s (see~\cite[Sec 4.1]{GRWZ20} for more details).
However, they do not bound the time to compute these exponents, which appears to be $O(bn)$ field operations in the worst-case.

The subvector proof can be verified in $O(b)$ time using $(g_2^{\alpha^{(n+1) - i}})_{i\in I}$ from \vrk as:
\begin{align}
e\left(c, \prod_{i\in I}\left(g_2^{\alpha^{(n+1)-i}}\right)^{t_i}\right) &= e(\pi_I, g_2) \left(g_T^{\alpha^{n+1}}\right)^{\sum_{i\in I} v_i t_i}\Leftrightarrow\\
e\left(c, g_2^{\sum _{i\in I}t_i \alpha^{(n+1)-i}}\right) &= e\left(\prod_{i\in I} \pi_i^{t_i}, g_2\right) g_T^{\alpha^{n+1}\sum_{i\in I} v_i t_i}\Leftrightarrow\\
 &= e\left(\prod_{i\in I} \pi_i^{t_i}, g_2\right) e\left(g_1^{\alpha^{n+1}\sum_{i\in I} v_i t_i}, g_2\right)\Leftrightarrow\\
 &= e\left(\prod_{i\in I} \pi_i^{t_i} \cdot g_1^{\alpha^{n+1}\sum_{i\in I} v_i t_i},g_2\right)\Leftrightarrow\\
 &= e\left(\prod_{i\in I} \pi_i^{t_i} \cdot \prod_{i\in I} g_1^{\alpha^{n+1} v_i t_i},g_2\right)\Leftrightarrow\\
 &= e\left(\prod_{i\in I} \left(\pi_i \cdot g_1^{\alpha^{n+1} v_i}\right)^{t_i},g_2\right)
\end{align}
Recall that $\pi_i= (c / g_1^{v_i \alpha^i})^{\alpha^{(n+1) - i}}$.
\begin{align}
e\left(c, g_2^{\sum _{i\in I}t_i \alpha^{(n+1)-i}}\right) &= e\left(\prod_{i\in I} \left((c / g_1^{v_i \alpha^i})^{\alpha^{(n+1) - i}} \cdot g_1^{\alpha^{n+1} v_i}\right)^{t_i},g_2\right)\Leftrightarrow\\
 &= e\left(\prod_{i\in I} \left((c / g_1^{v_i \alpha^i}) \cdot g_1^\frac{\alpha^{n+1} v_i}{\alpha^{(n+1) - i}}\right)^{t_i \alpha^{(n+1) - i}},g_2\right)\Leftrightarrow\\
 &= e\left(\prod_{i\in I} \left((c / g_1^{v_i \alpha^i}) \cdot g_1^{v_i\alpha^i}\right)^{t_i \alpha^{(n+1) - i}},g_2\right)\Leftrightarrow\\
 &= e\left(\prod_{i\in I} c^{t_i \alpha^{(n+1) - i}},g_2\right)\Leftrightarrow\\
 &= e\left( c^{\sum_{i\in I}t_i \alpha^{(n+1) - i}},g_2\right)\Leftrightarrow\\
 &= e\left( c,g_2^{\sum_{i\in I}t_i \alpha^{(n+1) - i}}\right)
\end{align}

\parhead{Aggregating Proofs and Precomputing All Proofs.}
A subvector proof requires $b$ hash computations and an $O(b)$-sized multi-exponentiation and thus takes $O(b)$ time.
Precomputing all proofs efficiently is not discussed.
Naively, it can be done in $O(n^2)$ time.

\subsection{Complexity of our Lagrange-based aSVC from \cref{s:asvc:from-kzg}}
\label{s:complexity-lagrange-asvc}
Our scheme builds upon previous VCs using KZG commitments~\cite{CDHK15, KZG10a}.
Since we give its full algorithmic description in \cref{s:asvc:from-kzg:algorithms}, this section will be briefer than previous ones.

\parhead{Public Parameters.}
The proving key, verification key and $i$th update key are $O(n), O(b)$ and $O(1)$-sized, respectively.
Similar to \cref{s:complexity-tcz}, $n$ is assumed to be a power of two, and $\lagr_i(X)=\prod_{j\in [0,n),j\ne i} \frac{X-\omega^j}{\omega^i - \omega^j}$ where $\omega$ is a primitive $n$th root of unity~\cite{vG13ModernCh8}.

\parhead{Commitment.}
A commitment is $c=\prod_{i\in[0,n)} \ell_i^{v_i}=g^{\phi(\tau)}$ where $\phi(X)=\sum_{i\in [0,n)} \lagr_i(X) v_i$ and $\phi(\omega^i) = v_i$.
% and can be computed with $O(n\log^2{n})$ field operations (see \cref{s:prelim:interpolation}) and $O(n)$ exponentiations.
If any vector element $v_j$ changes to $v_j + \delta$, the commitment can be updated in $O(1)$ time using as $c' = c \cdot (\upk_j)^{\delta} = c\cdot (\ell_j)^{\delta}$.

\parhead{Proofs for a $v_i$.}
\label{s:complexity-lagrange-asvc:proof}
A proof for $v_i$ is:
\begin{align}
\pi_i = g^{\frac{\phi(\tau)-v_i}{\tau - \omega^i}} = g^{q_i(\tau)}
\end{align}
However, note that:
\begin{align}
\frac{\phi(\tau)-\phi(\omega^i)}{\tau - \omega^i}
 &= \frac{\sum_{j\in [0,n)} \lagr_j(\tau) v_j - v_i}{\tau - \omega^i}\\
 &= \frac{\sum_{j\in [0,n)\setminus \{i\}} \lagr_j(\tau) v_j}{\tau - \omega^i} + \frac{\lagr_i(\tau)v_i - v_i}{\tau - \omega^i}\\
  &= \sum_{j\in [0,n)\setminus \{i\}}v_j\frac{\lagr_j(\tau)}{\tau - \omega^i} + v_i\frac{\lagr_i(\tau) - 1}{\tau - \omega^i}
\end{align}
Recall from \cref{s:asvc:from-kzg:updating-proofs} that (1) the $i$th update key contains a KZG commitment $u_i$ to $\frac{\lagr_i(\tau) - 1}{\tau - \omega^i}$ and that (2) the $a_i$'s and $a_j$'s from $\upk_i$ and $\upk_j$ can be used to compute in $O(1)$ time a KZG commitment $u_{i,j}$ to $\frac{\lagr_j(\tau)}{\tau - \omega^i}$.
(Note that the partial fraction decomposition only requires evaluating a degree-1 polynomial at two points. Also, computing $A'(\omega^j)$ can be done in $O(1)$ time as explained in \cref{app:xn-1-derivative}.)
Thus, the proof $\pi_i$ can be computed in $O(n)$ field operations and $O(n)$ exponentiations as:
\begin{align}
\pi_i &= g^{q_i(\tau)}=\prod_{j\in [0,n)\setminus \{i\}} \left(u_{i,j}\right)^{v_j} \cdot \left(u_i\right)^{v_i}
\end{align}

The proof can be verified in $O(1)$ time using $g^\tau$ from the \vrk by computing two pairings:
\begin{align}
e(c/g^{v_i}, g) = e(\pi_i, g^{\tau}/g^{\omega^i})
\end{align}

\parhead{Proof Updates.}
If any vector element $v_j,j\ne i$ changes to $v_j + \delta$, the proof $\pi_i$ can be updated in $O(1)$ time using $a_i,a_j$ from $\upk_i,\upk_j$.
First, one computes $u_{i,j}$ in $O(1)$ time as described in the previous paragraph.
Then, one updates $\pi_i' = \pi_i \cdot \left(u_{i,j}\right)^{\delta}$ in $O(1)$ time.
This new $\pi_i'$ will verify against the updated $c'$ commitment defined earlier.
If $v_i$ changes to $v_i + \delta$, the proof $\pi_i$ is updated in $O(1)$ time using $u_i$ from $\upk_i$ as $\pi_i' = \pi_i \cdot \left(u_i\right)^{\delta}$ (see \cref{s:asvc:from-kzg:updating-proofs}).

\parhead{Subvector Proofs for $(v_i)_{i\in I}$}
% TODO: Is it possible to compute a batch proof from the \ell_i's and/or \upk_i's (just like normal proofs can be computed from these)?
We use the same style of subvector proofs as in \cref{s:complexity-kzg}.
Thus, the subvector proving time is $O(n\log{n}+b\log^2{b})$ and the subvector proof verification time is $O(b\log^2{b})$ time.

\parhead{Aggregating Proofs and Precomputing All Proofs.}
Aggregating all proofs $(\pi_i)_{i\in I}$ requires computing coefficients $c_i=1/A'_I(\omega^i),\forall i \in I$ using partial fraction decomposition (see \cref{s:asvc:from-kzg:aggregating-proofs}).
This can be done by (1) computing $A_I(X)=\prod_{i \in I} (X-\omega^i)$ in $O(b\log^2{b})$ field operations, (2) computing its derivative $A'_I(X)$ in $O(b)$ field operations and (3) evaluating $A'_I(X)$ at all $(\omega^i)_{i\in I}$ using a multipoint evaluation in $O(b\log^2{b})$ field operations~\cite{vG13ModernCh10}.
Then, the subvector proof can be aggregated with $O(b)$ exponentiations as:
\begin{align}
\pi_I = \prod_{i\in I}\pi_i^{c_i}
\end{align}
Thus, aggregation takes $O(b\log^2{b})$ time.

Finally, precomputing all proofs can be done efficiently in $O(n\log{n})$ time using the FK technique~\cite{FK20}.

\parhead{Slower Commitment Time for Faster (Subvector) Proofs.}
When comitting to a vector, we can use the FK technique to precompute all $n$ proofs in $O(n\log{n})$ time and store them as \textit{auxiliary information}.
Then, we can serve any individual proof $\pi_i$ in $O(1)$ time and any subvector proof in $O(b\log^2{b})$ time by aggregating it from the $\pi_i$'s.

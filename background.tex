\section{Preliminaries}
\label{s:prelim}

\parhead{Notation.}
$\lambda$ is our security parameter.
$\G_1,\G_2$ are groups of prime order $p$ endowed with a \textit{pairing} $e : \G_1 \times \G_2 \rightarrow \G_T$\ifNotCameraReady~\cite{MVO91,Joux00}\fi.
(We assume symmetric pairings where $\G_1=\G_2$ for simplicity of exposition.)
$\Gho$ is a hidden-order group.
We use multiplicative notation for all groups.
$\omega$ is a primitive $n$th root of unity in $\Zp$~\cite{vG13ModernCh8}.
$\poly(\cdot)$ is any function upper-bounded by some univariate polynomial.
$\negl(\cdot)$ is any negligible function.
$\log{x}$ and $\lg{x}$ are shorthand for $\log_2{x}$.
$[i,j] = \{i,i+1,\dots,j-1,j\}$, $[0,n) =[0,n-1]$ and $[n]=[1,n]$.
$\vect{v}=(v_i)_{i\in[0,n)}$ is a vector of size $n$ with elements $v_i\in \Zp$.

\ifCameraReady
\parhead{Lagrange Interpolation.}
\else
\subsection{Lagrange Polynomial Interpolation}
\label{s:prelim:interpolation}
\fi
Given $n$ pairs $(x_i, y_i)_{i\in[0,n)}$, we can find or \textit{interpolate} the \textit{unique} polynomial $\phi(X)$ of degree $<n$ such that $\phi(x_i) = y_i, \forall i\in[0,n)$ using \textit{Lagrange interpolation}\ifNotCameraReady~\cite{BT04}\fi\xspace in $O(n\log^2{n})$ time~\cite{vG13ModernCh10} as
$\phi(X) = \sum_{i\in[0,n)} \lagr_i(X) y_i$, where $\lagr_i(X) = \prod_{\substack{j\in [0,n),j\ne i}}\frac{X-x_j}{x_i-x_j}$.
Recall that a \textit{Lagrange polynomial} $\lagr_i(X)$ has the property that $\lagr_i(x_i) = 1$ and $\lagr_i(x_j)=0, \forall i,j\in[0,n)$ with $j \ne i$.
Note that $\lagr_i(X)$ is defined in terms of the $x_i$'s which, throughout this paper, will be either $(\omega^i)_{i\in[0,n)}$ or $(\omega^i)_{i\in I}, I\subset [0,n)$.

\subsection{KZG Polynomial Commitments}
\label{s:prelim:polycommit:kzg}

Kate, Zaverucha and Goldberg (KZG) proposed a \textit{constant-sized} commitment scheme for degree $n$ polynomials $\phi(X)$.
Importantly, an \textit{evaluation proof} for any $\phi(a)$ is constant-sized and constant-time to verify; it does not depend in any way on the degree of the committed polynomial.
KZG requires public parameters $(g^{\tau^i})_{i\in[0,n]}$, which can be computed via a decentralized MPC protocol~\cite{BGM17} that hides the \textit{trapdoor} $\tau$.
KZG is computationally-hiding under the discrete log assumption and computationally-binding under $n$-SDH~\cite{BB08j}.

\parhead{Committing.}
Let $\phi(X)$ denote a polynomial of degree $d\le n$ with coefficients $c_0, c_1, \dots, c_d$ in $\Zp$.
A KZG commitment to $\phi(X)$ is a single group element $C = \prod_{i=0}^d {\left(g^{\tau^i}\right)^{c_i}} = g^{\sum_{i=0}^d c_i \tau^i} = g^{\phi(\tau)}$.
Committing to $\phi(X)$ takes $\Theta(d)$ time.

\parhead{Proving One Evaluation.}
To compute an \textit{evaluation proof} that $\phi(a) = y$, KZG leverages the polynomial remainder theorem, which says $\phi(a) = y \Leftrightarrow \exists q(X)\ \text{such that}\ \phi(X) - y = q(X)(X-a)$.
The proof is just a KZG commitment to $q(X)$: a single group element $\pi=g^{q(\tau)}$.
Computing the proof takes $\Theta(d)$ time.
To verify $\pi$, one checks (in constant time) if
$
e(C / g^y, g)            = e(\pi, g^{\tau}/g^a) \Leftrightarrow
e(g^{\phi(\tau) - y}, g) = e(g^{q(\tau)},g^{\tau-a})\Leftrightarrow
\ifNotCameraReady
e(g,g)^{\phi(\tau)-y}   = e(g,g)^{q(\tau)(\tau-a)} \Leftrightarrow
\fi
{\phi(\tau)-y}          = q(\tau)(\tau-a)
$.

\parhead{Proving Multiple Evaluations.}
Given a set of points $I$ and their evaluations $\{\phi(i)\}_{i\in I}$, KZG can prove all evaluations with a constant-sized \textit{batch proof} rather than $|I|$ individual proofs.
The prover computes an \textit{accumulator polynomial} $a(X)=\prod_{i\in I} (X-i)$ in $\Theta(|I|\log^2{|I|})$ time and computes $\phi(X)/a(X)$ in $\Theta(d\log{d})$ time, obtaining a quotient $q(X)$ and remainder $r(X)$.
The batch proof is $\pi_I=g^{q(\tau)}$.
To verify $\pi_I$ and $\{\phi(i)\}_{i\in I}$ against $C$, the verifier first computes $a(X)$ from $I$ and interpolates $r(X)$ such that $r(i)=\phi(i), \forall i \in I$ in $\Theta(|I|\log^2{|I|})$ time\ifNotCameraReady\xspace(see \cref{s:prelim:interpolation})\fi.
Next, she computes $g^{a(\tau)}$ and $g^{r(\tau)}$.
Finally, she checks if $e(C / g^{r(\tau)}, g) = e(g^{q(\tau)}, g^{a(\tau)})$.
We stress that batch proofs are only useful when $|I| \le d$.
Otherwise, if $|I| > d$, the verifier can interpolate $\phi(X)$ directly from the evaluations, which makes verifying any $\phi(i)$ trivial.

\subsection{Account-based Stateless Cryptocurrencies}
\label{s:prelim:stateless-cryptocurrency}

In a stateless cryptocurrency based on VCs~\cite{CPZ18}, there are \textit{miners} running a permissionless consensus algorithm~\cite{Nakamoto08} and \textit{users}, numbered from $0$ to $n-1$ who have \textit{accounts} with a \textit{balance} of coins.
($n$ can be $\infty$ if the VC is unbounded.)
For simplicity of exposition, we do not give details on the consensus algorithm, on transaction signature verification nor on monetary policy.
\ifNotCameraReady
These all remain the same as in previous stateful cryptocurrencies.
\fi

\parhead{The (Authenticated) State.}
The \textit{state} is an \textit{authenticated data structure (ADS)} mapping each user $i$'s \textit{public key} to their account balance $\bal_i$.
(In practice, the mapping is also to a \textit{transaction counter} $c_i$, which is necessary to avoid transaction replay attacks.
We address this in \cref{s:discussion:txn-counters}.)
Importantly, miners and users are \textit{stateless}: they do not store the state, just its \textit{digest} $d_t$ at the latest block $t$ they are aware of.
Additionally, each user $i$ stores a proof $\pi_{i,t}$ for their account balance that verifies against $d_t$.
%Note that some miners and/or users might be out of sync and have an earlier digest $d_{t-\Delta}$.

\parhead{Miners.}
Despite miners being stateless, they can still validate transactions, assemble them into a new \textit{block}, and propose that block.
Specifically, a miner can verify every new transaction spends valid coins by checking the sending user's balance against the latest digest $d_t$.
This requires each user $i$ who sends coins to $j$ to include her proof $\pi_{i,t}$ in her transaction.
Importantly, user $i$ should not have to include the recipient's proof $\pi_{j,t}$ in the transaction, since that would require interacting with \textit{proof-serving nodes} (see \cref{s:stateless-cryptocurrency:proof-serving-nodes})

Once the miner has a set $V$ of valid transactions, he can use them to create the next block ${t+1}$ and propose it.
The miner obtains this new block's digest $d_{t+1}$ by ``applying'' all transactions in $V$ to $d_t$.
When other miners receive this new block ${t+1}$, they can validate its transactions from $V$ against $d_t$ and check that the new digest $d_{t+1}$ was produced correctly from $d_t$ by ``reapplying'' all the transactions from $V$.

\parhead{Users.}
When creating a transaction \tx for block $t+1$, user $i$ includes her proof $\pi_{i,t}$ for miners to verify she has sufficient balance.
% ...and her counter is correct.
When she sees a new block ${t+1}$, she can update her proof $\pi_{i,t}$ to a new proof $\pi_{i,t+1}$, which verifies against the new digest $d_{t+1}$.
For this, she will look at all changes in balances $(j, \Delta\bal_j)_{j\in J}$, where $J$ is the set of users with transactions in block ${t+1}$, and ``apply'' those changes to her proof.
%(Similarly, when a miner sees a transaction from a user with a proof $\pi_{i,t-\Delta}$ w.r.t. to an earlier digest, the miner can update that proof to $\pi_{i,t}$ by keeping a window of the last $\Delta+1$ blocks and its transactions and applying those to the users proof.)
Similarly, miners can also update proofs of pending transactions which did not make it in block $t$ and now need a proof w.r.t. $d_{t+1}$

Users assume that the consensus mechanism produces correct blocks.
As a result, they do \textit{not} need to verify transactions in the block; they only need to update their own proof.
Nonetheless, since block verification is stateless and fast, users could easily participate as block validators, should they choose to.